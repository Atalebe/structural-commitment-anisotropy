\documentclass[11pt]{article}

\usepackage{amsmath,amssymb}
\usepackage{geometry}
\usepackage{setspace}
\usepackage{graphicx}
\usepackage{hyperref}
\geometry{margin=1in}
\onehalfspacing

\title{\textbf{Research Logbook: Directional Signatures of Structural Commitment}}
\author{Stephen Atalebe}
\date{\today}

\begin{document}

\maketitle

\section*{Purpose of this Logbook}

This document serves as the operational blueprint for a simulation-driven investigation into whether accumulated structural commitment can induce measurable late-time departures from statistical isotropy without violating the cosmological principle as an initial condition.

The emphasis is deliberately computational: simulations lead, theory follows.

\vspace{0.5cm}

\section*{Working Title}

\textbf{Directional Signatures of Structural Commitment: Testing Late-Time Isotropy in Non-Markovian Cosmology}

\vspace{0.5cm}

\section*{Draft Abstract}

The cosmological principle assumes statistical isotropy and homogeneity on large scales. However, nonlinear structure formation proceeds unevenly across cosmic environments, raising the possibility that accumulated structural history may introduce weak directional signatures in effective cosmological observables.

We present a simulation-driven investigation into whether direction-dependent structural commitment can generate hemispherical asymmetries or dipole-like modulations in effective expansion and growth. Using toy models, environment-partitioned N-body analyses, and synthetic observer reconstructions, we test whether late-time constraint accumulation produces observable anisotropy while remaining consistent with primordial isotropy.

This work establishes a computational pathway for evaluating history-sensitive cosmological dynamics and provides the first targeted test of anisotropic signatures within non-Markovian expansion frameworks.

\vspace{0.5cm}

\section*{Core Scientific Question}

\begin{quote}
Can uneven accumulation of structural commitment generate observable late-time anisotropy without violating primordial statistical isotropy?
\end{quote}

\vspace{0.5cm}

\section*{Introduction Structure}

\subsection*{1. Cosmological Principle}
\begin{itemize}
\item Statistical isotropy and homogeneity.
\item Observational support from large-scale surveys.
\item Role as a foundational assumption in modern cosmology.
\end{itemize}

\subsection*{2. Motivating Tensions}
\begin{itemize}
\item Reported hemispherical asymmetries.
\item Dipole modulations.
\item Bulk flow discussions.
\end{itemize}

Tone requirement: neutral. These motivate investigation, not claims.

\subsection*{3. Non-Markovian Expansion}
If effective expansion depends on accumulated nonlinear structure, and structure forms unevenly, directional dependence may naturally emerge at late times.

\subsection*{4. Objective}
Test whether structural commitment induces measurable anisotropic signatures.


\section*{Strategic Positioning}

This project does not challenge primordial isotropy.

It tests whether anisotropy can emerge from nonlinear cosmic history.

Symmetry first. History later.

\vspace{0.5cm}

\section*{Execution Roadmap}

\begin{itemize}
\item Week 1–2: Toy model.
\item Week 3–6: Simulation slicing and kernel estimation.
\item Week 7+: Synthetic observer analysis.
\end{itemize}

Guiding principle:

\begin{quote}
Mechanism before interpretation.
\end{quote}

\vspace{0.5cm}

\section*{Future Spinoffs}

\begin{itemize}
\item Formal theory of anisotropic structural commitment.
\item Constraint geometry in cosmological phase space.
\item Observational dipole searches.
\end{itemize}

\section{Baseline Mechanism: Memory-Regulated Emergence of Anisotropy}

Before turning to large-scale simulations, it is useful to establish whether uneven structural histories are capable, in principle, of generating directional signatures in a history-dependent cosmology. 
To this end, we construct a minimal toy model designed not for realism but for dynamical transparency. 
The goal is to isolate the role of cosmological memory under controlled conditions and determine whether anisotropy can emerge without violating primordial isotropy.


\begin{figure*}[t]
\centering
\includegraphics[width=0.85\textwidth]{figures/structural_commitment.png}
\caption{
Toy model demonstration of emergent anisotropy from uneven structural commitment. 
\textbf{Left:} Region-dependent structural histories showing earlier nonlinear collapse in Region A. 
\textbf{Center:} Retarded memory response generated by convolution with a causal kernel. 
\textbf{Right:} Resulting dipole in the effective expansion rate. The signal grows during asymmetric structure formation and relaxes once both regions approach saturation, illustrating that late-time anisotropy can emerge without violating primordial isotropy.
}
\label{fig:toy_dipole}
\end{figure*}
\begin{figure}[t]
\centering
\includegraphics[width=0.65\textwidth]{figures/dipole_strength_vs_memory_horizon.png}
\caption{
Maximum dipole amplitude as a function of memory timescale. 
Increasing the memory horizon suppresses transient anisotropies by averaging over a longer structural history. 
This behavior suggests that extended cosmological memory may act as an isotropizing mechanism even in the presence of uneven nonlinear evolution.
}
\label{fig:dipole_memory}
\end{figure}
This result highlights a dual role for cosmological memory: finite horizons can amplify directional signatures, whereas extended horizons suppress them.
Cosmological memory plays a dual role: it transmits structural history while simultaneously regulating the magnitude of emergent anisotropy.Extended memory horizons may act as a natural isotropizing mechanism, suppressing transient directional imprints generated during uneven nonlinear evolution.
\begin{figure}[t]
\centering
\includegraphics[width=0.75\textwidth]{figures/dipole_evolution_power_law_memory.png}
\caption{
Dipole evolution generated by power-law memory kernels with varying exponent $\beta$. 
Larger $\beta$ values correspond to shorter effective memory and produce deeper transient anisotropies, while smaller $\beta$ values retain a longer structural history and suppress the dipole amplitude through temporal averaging. 
This behavior indicates that the magnitude of emergent anisotropy depends not only on uneven structure formation but also on the temporal weighting encoded in the cosmological memory kernel.
}
\label{fig:dipole_powerlaw}
\end{figure}
Figure~\ref{fig:dipole_powerlaw} demonstrates that anisotropy amplitude is strongly regulated by the temporal structure of the memory kernel. 
Short-memory kernels amplify instantaneous differences in structural commitment, producing deeper dipoles, whereas long-tailed kernels integrate over a broader evolutionary history and thereby suppress directional contrasts. 
This suggests that extended cosmological memory may function as a stabilizing mechanism, preserving large-scale isotropy even in the presence of uneven nonlinear growth.

The magnitude of emergent anisotropy is jointly controlled by structural divergence and memory decay, with cosmological memory acting as a regulatory mechanism that prevents runaway directional bias.

\subsection{Scaling of Anisotropy with Memory Decay}
\begin{figure}[t]
\centering
\includegraphics[width=0.7\textwidth]{max_dipole_vs_beta.png}
\caption{
Maximum dipole amplitude as a function of the power-law memory exponent $\beta$. 
Larger $\beta$ values correspond to faster kernel decay and shorter effective memory, leading to stronger transient anisotropies. 
Conversely, long-tailed kernels suppress the dipole through temporal averaging. 
This scaling relation indicates that the persistence and magnitude of directional signatures are directly regulated by the temporal structure of cosmological memory.
}
\label{fig:max_dipole_beta}
\end{figure}
Figure~\ref{fig:max_dipole_beta} reveals a clear scaling between anisotropy amplitude and memory decay rate. 
As the kernel becomes more short-lived, instantaneous differences in structural commitment are amplified, producing stronger dipoles. 
In contrast, extended memory horizons integrate over a broader evolutionary history and act to suppress directional contrasts. 
This behavior supports the interpretation of cosmological memory as a regulator of large-scale isotropy.

\subsection{Regulation of Anisotropy by Memory Decay}
\begin{figure}[t]
\centering
\includegraphics[width=0.72\textwidth]{figures/dipole_strenght_vs_memory_decay.png}
\caption{
Maximum dipole amplitude as a function of the power-law memory exponent $\beta$ for moderately separated collapse histories. 
Faster kernel decay (larger $\beta$) amplifies instantaneous structural contrasts, producing stronger anisotropies, while long-tailed memory suppresses the signal through temporal averaging. 
The monotonic and saturating behavior indicates that memory decay regulates the magnitude of emergent directional signatures.
}
\label{fig:max_dipole_beta}
\end{figure}
\begin{figure}[t]
\centering
\includegraphics[width=0.72\textwidth]{figures/Figure_4 Dipole strenght vs memory decay rate.png}
\caption{
Same analysis as Fig.~\ref{fig:max_dipole_beta}, but for more widely separated collapse epochs. 
Although the overall dipole amplitude increases, the scaling relation remains monotonic and bounded, demonstrating that the anisotropy mechanism is robust under significant structural divergence.
}
\label{fig:max_dipole_beta_wide}
\end{figure}
Figures~\ref{fig:max_dipole_beta} and \ref{fig:max_dipole_beta_wide} reveal a consistent scaling between anisotropy amplitude and memory decay rate across distinct structural regimes. 
While greater separation in collapse histories elevates the overall dipole, the signal remains bounded and exhibits clear saturation at large $\beta$. 
This behavior suggests that cosmological memory acts as a regulatory mechanism, jointly with structural divergence, to constrain the magnitude of emergent directional bias.
We emphasize that this model is not intended to represent a realistic cosmological environment, but rather to establish a baseline mechanism against which simulation-based results can be interpreted.
Having established that memory-regulated dynamics can generate bounded anisotropy under controlled conditions, we now turn to large-scale structure simulations to assess whether comparable signatures arise in realistic nonlinear environments.
\section{Directional Structure in Cosmological Simulations}
\begin{figure}[t]
\centering
\includegraphics[width=0.7\textwidth]{figures/structure_dipole_histogram.png}
\caption{
Distribution of structural dipole amplitudes measured across randomly sampled axes for a synthetic observer within the simulation volume. 
The spread reflects intrinsic directional variations in nonlinear structure formation.
}
\label{fig:structure_dipole}
\end{figure}

\subsection{Structural Isotropy Baseline}

Before introducing history-dependent dynamics, we quantify the intrinsic directional imbalance of the halo distribution in the TNG300-1 simulation at $z=0$. Establishing this baseline is essential for distinguishing genuine dynamical anisotropy from geometric fluctuations arising in finite cosmological volumes.

We construct unit vectors from the simulation barycenter to each subhalo and evaluate the projected mean along randomly oriented axes $\hat{n}$:

\begin{equation}
D(\hat{n}) = \left| \left\langle \hat{r}_i \cdot \hat{n} \right\rangle \right|.
\end{equation}

Sampling 500 random directions yields

\begin{align}
\langle D \rangle &= 1.08 \times 10^{-2}, \\
\sigma_D &= 6.08 \times 10^{-3}, \\
D_{\max} &= 2.17 \times 10^{-2}.
\end{align}

These values indicate that large-scale structure alone produces directional imbalances at the percent level within a finite $\sim300\,h^{-1}\mathrm{Mpc}$ volume. The simulation therefore remains statistically consistent with isotropy, while providing a quantitative noise floor against which any memory-induced directional expansion must be evaluated.

This baseline ensures that subsequent anisotropy measurements probe dynamical effects rather than geometric artifacts of the halo distribution.

\subsection{Structural Dipole Stability and Directional Persistence}
\label{sec:structural_dipole}

Before attempting to infer any memory-induced modulation of the expansion field, it is essential to characterise the intrinsic anisotropy of the structural source itself. We therefore measure the mass-weighted dipole of the subhalo distribution in the TNG300-1 simulation across multiple snapshots spanning late cosmic time.

The dipole vector is defined as

\begin{equation}
\mathbf{D} = \frac{1}{\sum_i M_i} \sum_i M_i \, \hat{\mathbf{r}}_i ,
\end{equation}

where $M_i$ is the subhalo mass and $\hat{\mathbf{r}}_i$ is the unit vector from the box centre to each subhalo.

\subsubsection{Mass-threshold stability}

To test whether the signal is dominated by low-mass sampling noise or by the gravitational backbone of the cosmic web, we recompute the dipole under progressively stricter mass cuts at snapshot 72.

\begin{center}
\begin{tabular}{lc}
\hline
Mass Selection & Dipole Amplitude \\
\hline
No cut & 0.074 \\
$>10^{10}\, M_\odot/h$ & 0.082 \\
$>10^{11}\, M_\odot/h$ & 0.091 \\
$>10^{12}\, M_\odot/h$ & 0.120 \\
\hline
\end{tabular}
\end{center}

The monotonic increase of the dipole with halo mass indicates that the anisotropy is not driven by numerical noise or small-scale fluctuations, but is instead anchored in the most massive collapsed structures.

\subsubsection{Time evolution}

We next compute the dipole amplitude across snapshots:

\begin{center}
\begin{tabular}{lc}
\hline
Snapshot & Dipole Amplitude \\
\hline
33 & 0.030 \\
40 & 0.031 \\
50 & 0.037 \\
59 & 0.048 \\
67 & 0.033 \\
72 & 0.074 \\
78 & 0.064 \\
91 & 0.061 \\
\hline
\end{tabular}
\end{center}

The dipole strengthens toward lower redshift, consistent with the nonlinear amplification of large-scale structure.

\subsubsection{Directional coherence}

To assess whether this anisotropy reflects transient fluctuations or persistent structural organisation, we compute the cosine similarity between dipole vectors across snapshots. Typical values exceed $0.9$, indicating remarkable directional stability over several gigayears.

Such persistence implies that the simulation volume develops a preferred collapse axis that remains coherent through late-time evolution.

\subsubsection{Interpretation}

These measurements establish three important facts:

\begin{itemize}
\item The structural field is not perfectly isotropic within a finite cosmological volume.
\item The anisotropy is dominated by massive halos rather than low-mass noise.
\item The dipole direction remains stable across cosmic time.
\end{itemize}

This combination is precisely the regime in which a history-dependent kernel could, in principle, transmit structural asymmetries into the effective expansion field.

We therefore emphasise that the structural dipole itself is not evidence for non-Markovian dynamics. Rather, it provides the necessary input for testing whether the expansion responds causally to anisotropic collapse histories.

\subsection{Structural dipole null baseline and time series (TNG300-1)}
\label{subsec:structural_dipole_baseline}

\paragraph{Aim.}
Before looking for any memory-induced directional signal, I first measured how large a purely stochastic structural dipole is in an isotropic $\Lambda$CDM volume with TNG300-1 initial conditions. The goal is to treat this as the noise floor for any later expansion dipole.

\paragraph{Null test: random hemisphere baseline at $z=0$.}
Using \texttt{structural\_isotropy\_baseline.py} on TNG300-1 (\texttt{BASE\_PATH = data/TNG300}, snapshot $99$ with
\texttt{data/TNG300/groups\_099} $\rightarrow$ \texttt{/mnt/g/TNG300-1/groupcat\_99}),
I loaded all subhalos with positions and masses:
\[
N_{\rm sub} = 14{,}485{,}709 \, .
\]
After recentring to the box midpoint and applying periodic wrapping, I defined a mass–weighted hemisphere dipole along a unit vector $\hat{n}$ as
\begin{equation}
D(\hat{n}) \equiv 
\frac{M_{+}(\hat{n}) - M_{-}(\hat{n})}{M_{+}(\hat{n}) + M_{-}(\hat{n})} \, ,
\end{equation}
where $M_{\pm}$ are the total subhalo masses in the two hemispheres defined by ${\rm sign}(\hat{r}\cdot\hat{n})$.
I then drew $N_{\rm rand} = 2000$ random unit vectors $\hat{n}$ (uniformly on the sphere) and measured $|D(\hat{n})|$ for each.

The resulting null distribution of structural dipole amplitudes at $z\simeq 0$ is:
\begin{align}
\mu_D      &\simeq 0.0422 \, , \\
\sigma_D   &\simeq 0.0216 \, , \\
D_{\rm max}&\simeq 0.0975 \, , \\
D_{\rm min}&\simeq 3.6\times 10^{-5} \, .
\end{align}
This means a TNG300-1 box of side $L = 205\,{\rm Mpc}/h$ naturally produces hemisphere–contrast structural dipoles of order
\[
|D| \sim 4\% \quad \text{with}\quad \mathcal{O}(2\%)\ \text{scatter}
\]
purely from Gaussian cosmic variance. Any later “signal” in the expansion dipole has to be interpreted relative to this baseline, not relative to zero. The null distribution is saved as
\texttt{structural\_dipole\_null\_distribution.npy}, and its histogram as
\texttt{fig\_structural\_dipole\_null.png}.

\paragraph{Mass–threshold sanity check at $z \simeq 0.7$ (snapshot 72).}
As a quick robustness check, I repeated the hemisphere dipole measurement at snapshot $72$ (\texttt{/mnt/g/TNG300-1/groupcat\_072}) and imposed successive mass cuts on the subhalos. The behaviour is monotonic:
\begin{align}
\text{no cut}       &:\quad |D| \simeq 0.0744, \\
M > 10^{10}\,M_\odot/h &:\quad |D| \simeq 0.0825, \\
M > 10^{11}\,M_\odot/h &:\quad |D| \simeq 0.0912, \\
M > 10^{12}\,M_\odot/h &:\quad |D| \simeq 0.1198.
\end{align}
The more massive end of the hierarchy is more anisotropically distributed along the same preferred axis. This is good news: any virialisation-based or massive–halo-based choice of structural source $\Sigma(\hat{n},t)$ will naturally inherit a stronger dipole than the full population, without flipping direction. The corresponding summary plot is stored as \texttt{fig\_mass\_thresholds.png}.

\paragraph{Time series and directional coherence.}
I then measured the structural dipole as a function of time using
\texttt{dipole\_time\_series.py} on the late-time snapshot set
\[
\{33,40,50,59,67,72,78,91\},
\]
always using the full subhalo population in the corresponding
\texttt{groupcat\_XXX} folders under \texttt{/mnt/g/TNG300-1}.
For each snapshot $s$, the code computes a mass–weighted dipole vector $\vec{D}_s$ and its amplitude $|\vec{D}_s|$. The amplitudes are
\begin{equation}
|\vec{D}_s| =
\{0.0296,\ 0.0310,\ 0.0370,\ 0.0485,\ 0.0329,\ 0.0744,\ 0.0640,\ 0.0611\},
\end{equation}
for snapshots
$\{33,40,50,59,67,72,78,91\}$ respectively.
Numerically, the structural dipole is at the few–per–cent level throughout, i.e.\ always within the $1$–$2\sigma$ range of the null distribution measured at $z=0$.

More important than the raw amplitudes is the directional stability. Normalising each dipole vector to a unit vector $\hat{D}_s$, I formed the cosine matrix
\[
C_{ij} = \hat{D}_i \cdot \hat{D}_j ,
\]
whose entries are
\[
C_{ij} \approx
\begin{pmatrix}
1.    & 0.978 & 0.830 & 0.959 & 0.990 & 0.942 & 0.995 & 0.873 \\
0.978 & 1.    & 0.881 & 0.984 & 0.952 & 0.963 & 0.994 & 0.897 \\
0.830 & 0.881 & 1.    & 0.949 & 0.846 & 0.969 & 0.850 & 0.595 \\
0.959 & 0.984 & 0.949 & 1.    & 0.953 & 0.996 & 0.973 & 0.804 \\
0.990 & 0.952 & 0.846 & 0.953 & 1.    & 0.946 & 0.975 & 0.799 \\
0.942 & 0.963 & 0.969 & 0.996 & 0.946 & 1.    & 0.952 & 0.746 \\
0.995 & 0.994 & 0.850 & 0.973 & 0.975 & 0.952 & 1.    & 0.898 \\
0.873 & 0.897 & 0.595 & 0.804 & 0.799 & 0.746 & 0.898 & 1.
\end{pmatrix}.
\]
Off–diagonal entries are typically $\gtrsim 0.95$, and even the most widely separated snapshots retain $C_{ij}\sim 0.6$–$0.9$. In other words, the structural dipole direction is not a random walk; it is a coherent preferred axis that persists across the late-time snapshot sequence. The matrix and its visualisation are saved as
\texttt{dipole\_vectors.npy} and \texttt{fig\_dipole\_directions.png}.
The amplitude evolution and its comparison to the null baseline are plotted in
\texttt{fig\_dipole\_time\_series.png}.

\paragraph{Takeaway.}
At this point the structural side is calibrated. A TNG300-1 box of size $L=205\,{\rm Mpc}/h$ naturally carries a few–per–cent structural dipole with a strongly coherent direction in time, and this amplitude grows when I weight towards higher-mass subhalos. The next step is no longer to ask “is there a dipole?” but to measure how the expansion field responds to this anisotropic structural history, i.e.\ to extract the $H$–$\Sigma$ transfer function under the memory kernel inferred in the viscosity and scale–ladder papers.

\paragraph{Interpretation and role in the anisotropy test.}

Putting these pieces together, we conclude that the TNG300 box hosts a single, coherent large--scale structural mode whose amplitude grows towards low redshift and is traced more strongly by massive haloes. The raw hemispheric dipole amplitudes lie within $\lesssim 1.5\,\sigma$ of the random--orientation null distribution, so in isolation they are fully compatible with the expected scatter for a single $\Lambda$CDM realisation. The key signatures are instead the persistent directional coherence over $\sim 8$~Gyr and the monotonic increase of the signal with halo mass threshold. We therefore treat the measured structural dipole,
$d_\Sigma(z)\simeq 0.03$--$0.07$ over $0.2\lesssim z\lesssim 2$ and up to $d_\Sigma\simeq 0.12$ for $M_\mathrm{halo}>10^{12}~h^{-1}M_\odot$ at $z\simeq 0.5$, as a controlled input mode for the subsequent anisotropy experiment. In the next stage of the logbook we will measure the response of the bulk flow and the effective expansion rate to this mode, and compare the resulting transfer function to the expectations from the non--Markovian memory kernel calibrated in the companion analyses.
\begin{figure}
    \centering
    \includegraphics[width=0.48\textwidth]{figures/fig_structural_dipole_null.png}
    \caption{
        Null distribution of the structural dipole amplitude at $z=0$
        from random hemisphere splits of the TNG300-1 subhalo
        population. The vertical line marks the mean, with the shaded
        region indicating the one-sigma interval.
    }
    \label{fig:null_dipole_distribution}
\end{figure}
\begin{figure}
    \centering
    \includegraphics[width=0.48\textwidth]{figures/fig_dipole_time_series.png}
    \caption{
        Structural dipole amplitude as a function of snapshot, compared
        to the null mean and one-sigma band from the random-orientation
        baseline. The late-time growth reflects the nonlinear
        amplification of anisotropic structure within the finite
        simulation volume.
    }
    \label{fig:dipole_time_series}
\end{figure}
\begin{figure}
    \centering
    \includegraphics[width=0.48\textwidth]{figures/fig_mass_thresholds.png}
    \caption{
        Dipole amplitude at snapshot 72 as a function of minimum subhalo
        mass. The signal strengthens when restricting to more massive
        halos, indicating that the anisotropy is anchored in the
        high-mass skeleton of the cosmic web rather than low-mass
        sampling noise.
    }
    \label{fig:mass_thresholds}
\end{figure}
\begin{figure}
    \centering
    \includegraphics[width=0.48\textwidth]{figures/fig_dipole_directions.png}
    \caption{
        Cosine of the angles between structural dipole directions across
        snapshots. Values close to unity show that the preferred axis of
        collapse remains coherent over late cosmic time, providing the
        necessary condition for any memory kernel to transmit structural
        anisotropy into the effective expansion field.
    }
    \label{fig:dipole_direction_coherence}
\end{figure}
\subsection{Bulk flow, structural dipole, and kinematic artefacts}
\label{sec:bulkflow_alignment}

Having established the existence of a coherent structural dipole in TNG300, the next obvious sanity check was to ask whether this dipole is just a kinematic artefact of the simulation volume drifting through space. In other words: is the ``arrow of time'' vector simply the net velocity of the box, or is it an intrinsic orientation of the large scale potential.

For the same snapshot set used in the structural time series,
\[
\text{snapshots} = \{33, 40, 50, 59, 67, 72, 78, 91\},
\]
I computed the mass weighted bulk flow vector
\begin{equation}
\mathbf{v}_{\rm bulk}(t)
= \frac{\sum_i m_i \,\mathbf{v}_i}{\sum_i m_i},
\end{equation}
using all subhalos from the TNG300\_1 group catalogs. Here $m_i$ is \texttt{SubhaloMass} in the native Illustris units (in $10^{10} M_\odot/h$), and $\mathbf{v}_i$ is \texttt{SubhaloVel} in km s$^{-1}$. All catalogues were read directly from the folders
\texttt{groupcat\_033}, \texttt{groupcat\_040}, \dots, \texttt{groupcat\_091} under \texttt{/mnt/g/TNG300-1}, and I verified that each folder contains the expected set of
\texttt{fof\_subhalo\_tab\_XXX.*.hdf5} files.

For each snapshot I recorded the bulk flow amplitude
\[
V_{\rm bulk}(t) \equiv \big\lVert \mathbf{v}_{\rm bulk}(t) \big\rVert
\]
and compared the unit vector $\hat{\mathbf{v}}_{\rm bulk}(t)$ to the structural dipole direction $\hat{\mathbf{d}}_\Sigma(t)$ from the previous section via
\begin{equation}
\cos\theta(t) \;=\; \hat{\mathbf{d}}_\Sigma(t)\cdot \hat{\mathbf{v}}_{\rm bulk}(t),
\qquad
\theta(t) \;=\; \arccos\left[\cos\theta(t)\right].
\end{equation}
The main empirical facts are:

\begin{itemize}
  \item The bulk flow amplitude is small at all times. It grows from
  \[
  V_{\rm bulk} \approx 0.25~{\rm km\,s^{-1}} \quad (\text{snap }33)
  \]
  to
  \[
  V_{\rm bulk} \approx 15.5~{\rm km\,s^{-1}} \quad (\text{snap }91),
  \]
  with typical values of $\mathcal{O}(2$–$12)$ km s$^{-1}$ in between. On $\sim 200~h^{-1}{\rm Mpc}$ scales, this is negligible compared to the $200$–$300$ km s$^{-1}$ bulk flows seen in the real Universe. In the usual comoving simulation frame, the TNG300 volume is effectively at rest: the box is not secretly racing across the CMB sky, it is barely shuffling.

  \item The alignment between the structural dipole and the bulk flow is not locked. The cosine and angle for each snapshot span
  \begin{align*}
  \cos\theta &\simeq -0.41 \quad (\theta \simeq 114^\circ,~\text{snap }33),\\
  \cos\theta &\simeq  +0.74 \quad (\theta \simeq  42^\circ,~\text{snap }40),\\
  \cos\theta &\simeq  -0.23 \quad (\theta \simeq 103^\circ,~\text{snap }50),\\
  \cos\theta &\simeq  -0.17 \quad (\theta \simeq 100^\circ,~\text{snap }59),\\
  \cos\theta &\simeq  +0.36 \quad (\theta \simeq  69^\circ,~\text{snap }67),\\
  \cos\theta &\simeq  -0.60 \quad (\theta \simeq 127^\circ,~\text{snap }72),\\
  \cos\theta &\simeq  +0.10 \quad (\theta \simeq  84^\circ,~\text{snap }78),\\
  \cos\theta &\simeq  -0.05 \quad (\theta \simeq  93^\circ,~\text{snap }91).
  \end{align*}
  The sign flips several times, cycling through weak infall, weak outflow, and nearly orthogonal configurations. There is no simple picture in which the density dipole and the box velocity are locked into a single infall direction.

  \item The most important late time result is the near perfect orthogonality at $z \simeq 0.2$ (snapshot 91):
  \[
  \cos\theta \simeq -0.052 \quad\Rightarrow\quad \theta \simeq 93^\circ.
  \]
  By the time the structural dipole and its direction have settled, the residual bulk flow is almost exactly perpendicular to that direction. If the dipole were just a ``wake'' from a drifting box we would expect $\theta \approx 0^\circ$ or $180^\circ$ at late times, not $90^\circ$.
\end{itemize}

There is one clear dynamical event in this sequence. At $z \simeq 0.5$ (snapshot 72) the structural dipole amplitude peaks at
\[
d_\Sigma \simeq 0.074,
\]
the bulk flow jumps to $V_{\rm bulk} \simeq 10.8~{\rm km\,s^{-1}}$, and the alignment settles into a strong infall like configuration with $\cos\theta \simeq -0.60$ ($\theta \simeq 127^\circ$). This looks like a genuine large scale collapse or violent relaxation: a major rearrangement of mass that both maximises structural anisotropy and kicks the box into coherent motion. By $z \lesssim 0.4$ (snapshots 78 and 91), the bulk flow continues to grow but the alignment relaxes to almost orthogonal values, suggesting that the post collapse flows are predominantly transverse to the density gradient rather than simple radial infall.

Overall, these measurements do three jobs. First, they confirm that the structural dipole identified earlier is not a numerical artefact of a moving simulation frame: the net motion is tiny, and the late time bulk flow is orthogonal to the dipole. Second, they show that the dipole tracks genuine dynamical history inside the box, spiking in lockstep with a major collapse event around $z \simeq 0.5$. Third, they make it very clear that any memory induced expansion dipole must be separated from the kinematic dipole generated by bulk motion, not confused with it. The structural arrow of time is an intrinsic orientation of the cosmic web potential in this volume, not just the direction in which the box happens to drift.

\subsection{Implications for the memory kernel and the expansion dipole}
\label{sec:implications_memory}

The combined structural and kinematic measurements now give a clean picture of what the anisotropy experiment is really testing.

\paragraph{Three dynamical regimes.}
Looking at the time series together, the box passes through three qualitatively different regimes:

\begin{itemize}
  \item At early times $z \gtrsim 0.6$ (snapshots 33–67) the structural dipole amplitude is modest and the bulk flow is weak. The alignment angle oscillates between mild infall and mild outflow ($\cos\theta \sim \pm 0.2$–0.4). The large scale mode is still assembling.

  \item Around $z \simeq 0.5$ (snapshot 72) there is a clear collapse event. The structural dipole jumps to $d_\Sigma \simeq 0.074$, the bulk flow spikes to $V_{\rm bulk} \simeq 10.8~{\rm km\,s^{-1}}$, and the alignment swings to a strong infall configuration with $\cos\theta \simeq -0.60$. This is where the density field and the velocity field point most cleanly into the same structure.

  \item At late times $z \lesssim 0.4$ (snapshots 78 and 91) the structural dipole remains high ($d_\Sigma \sim 0.06$) and the bulk flow continues to grow to $V_{\rm bulk} \simeq 15.5~{\rm km\,s^{-1}}$, but the alignment relaxes to almost orthogonal values ($\cos\theta \approx 0$, $\theta \approx 90^\circ$). The flow becomes more shear like, with motion predominantly transverse to the main density gradient.
\end{itemize}

In the viscosity and scale ladder papers, the memory kernel $K(\Delta t)$ was measured from the correlation between a scalar structural source $\Sigma(t)$ and the global expansion proxy $\delta H^2(t)$. Here the same kernel must be applied in a directional setting:
\[
\delta H^2(t,\hat{\mathbf{n}}) \; \sim \;
\int {\rm d}\tau\, K(t-\tau)\,\Sigma(\tau,\hat{\mathbf{n}}),
\]
with $\Sigma(\tau,\hat{\mathbf{n}})$ traced by the structural dipole and higher multipoles, and with an additional kinematic dipole coming from the bulk flow.

\paragraph{Kinematic dipole versus memory dipole.}
From the bulk flow amplitudes, the purely kinematic contribution to an expansion dipole is of order
\[
\left. \frac{\delta H}{H} \right|_{\rm kin}
\;\sim\; \frac{V_{\rm bulk}}{c}
\;\sim\; 5\times 10^{-5}
\quad \text{for} \quad V_{\rm bulk} \simeq 15~{\rm km\,s^{-1}}.
\]
The structural dipole and the measured kernel from the companion work predict a memory induced expansion dipole of similar order,
\[
\left. \frac{\delta H}{H} \right|_{\rm mem}
\;\sim\; \text{few}\times 10^{-5}
\quad \text{for} \quad d_\Sigma \sim 0.07,
\]
so the two effects are naturally comparable. In other words, the simulation is operating in the regime where the non Markovian signal is not swamped, but it is not safely orders of magnitude above the kinematic floor either. This sets the standard of care for the next stage.

\paragraph{Next steps.}
Operationally, the anisotropy experiment reduces to three tasks:

\begin{enumerate}
  \item Construct a directional expansion proxy $\delta H^2(t,\hat{\mathbf{n}})$ at fixed time, using either radial velocities (a local Hubble fit in different hemispheres) or the same $\delta H^2$ estimator used in the viscosity paper but binned on the sky.

  \item Measure the angular dipole of $\delta H^2(t,\hat{\mathbf{n}})$ and compare it to the angular dipole of the structural source $\Sigma(t,\hat{\mathbf{n}})$ and to the bulk flow direction. In particular, at $z \simeq 0.5$ we expect a negative $\delta H^2$ excess in the dipole direction (infall into the forming structure), while at $z \lesssim 0.4$ we expect a more shear dominated pattern.

  \item Subtract the kinematic dipole associated with $\mathbf{v}_{\rm bulk}(t)$, which contributes a known $V_{\rm bulk}/c$ pattern to $\delta H^2(t,\hat{\mathbf{n}})$. The residual dipole, after this subtraction, is the genuine memory signal to be compared with the kernel based prediction.
\end{enumerate}

The structural dipole and its null distribution have therefore done their job. They have turned the problem from ``is there any anisotropy at all'' into a much sharper question: \emph{does the directional expansion respond to the measured large scale structural mode with the amplitude and phase predicted by the previously measured memory kernel, once ordinary kinematic effects are stripped away}. The bulk flow analysis above shows that this test is not contaminated by a trivial moving box artefact; what remains is a clean, if demanding, causal check of the non Markovian framework.




\subsection{Directional expansion response to the structural dipole}
\label{subsec:expansion_dipole}

Having established that the large–scale structural dipole is a genuine mode of the density field (Section~\ref{subsec:structural_dipole}) and that the simulation volume develops a modest coherent bulk flow of order $v_{\rm bulk} \sim 10$--$15\ {\rm km\,s^{-1}}$ at late times (Section~\ref{subsec:bulk_flow}), we now perform a first directional measurement of the effective expansion rate along the dipole axis.

For each snapshot in the sequence
\[
\{33,\,40,\,50,\,59,\,67,\,72,\,78,\,91\}
\]
we locate the corresponding TNG300 group catalogue folder
\texttt{groupcat\_XXX} and load all subhaloes with their comoving positions
$\vec{x}$, peculiar velocities $\vec{v}$ and masses $M_{\rm sub}$ from the
\texttt{fof\_subhalo\_tab\_XXX.*.hdf5} files. Positions are given in comoving
kpc$/h$, velocities in km\,s$^{-1}$, and masses in units of
$10^{10}\,M_\odot/h$. Unless otherwise stated we impose no lower mass cut, so
all resolved subhaloes contribute as tracers of the large–scale flow.

We work in a box–centred frame, with the origin at
\[
\vec{x}_{\rm c} = \frac{L}{2}(1,1,1), \qquad L = 205\ {\rm Mpc}/h,
\]
and define for each halo the comoving displacement vector
$\vec{r} = \vec{x} - \vec{x}_{\rm c}$ and the corresponding unit vector
$\hat{n} = \vec{r}/|\vec{r}|$. To avoid numerical noise from the immediate
centre we exclude objects with $|\vec{r}| < 1\ {\rm kpc}/h$. Before analysing
the radial flow we subtract the mass–weighted bulk velocity
$\vec{v}_{\rm bulk}(t)$ measured in Section~\ref{subsec:bulk_flow}, so that
\[
\vec{v}_{\rm corr} = \vec{v} - \vec{v}_{\rm bulk}(t),
\qquad
v_r = \vec{v}_{\rm corr}\cdot \hat{n}.
\]

As a crude but informative proxy for the local expansion rate we perform a
mass–weighted linear regression of the form
\begin{equation}
  v_r = \delta H(t)\,r_{\rm com},
\end{equation}
where $r_{\rm com} = |\vec{r}|$ is the comoving radius in Mpc$/h$ and
$\delta H(t)$ is obtained from a weighted least–squares fit with zero
intercept,
\begin{equation}
  \delta H(t)
  = \frac{\sum_i w_i\,r_{{\rm com},i}\,v_{r,i}}{\sum_i w_i\,r_{{\rm com},i}^2},
  \qquad w_i = M_{{\rm sub},i}.
\end{equation}
This fit is carried out in three configurations:
(i) using all halos in the box (``all–sky''), (ii) restricted to the
hemisphere aligned with the structural dipole direction $\hat{d}(t)$, and
(iii) restricted to the opposite hemisphere. The unit vector $\hat{d}(t)$ is
taken from the structural dipole measurement in
Section~\ref{subsec:structural_dipole}, where we evaluated the mass–weighted
centre–of–mass offset and its time evolution.

The hemispheres are defined by the sign of the projection
$\hat{n}\cdot\hat{d}(t)$,
\begin{equation}
  {\rm North:}\ \hat{n}\cdot\hat{d} \ge 0, \qquad
  {\rm South:}\ \hat{n}\cdot\hat{d} < 0,
\end{equation}
yielding three effective slopes
$\delta H_{\rm all}(t)$, $\delta H_{\rm N}(t)$ and $\delta H_{\rm S}(t)$.
We then construct a dipole–like expansion contrast
\begin{equation}
  \Delta H(t) \equiv \frac{1}{2}\,\big[\delta H_{\rm N}(t) - \delta H_{\rm S}(t)\big],
\end{equation}
which quantifies how much more (or less) the Universe is effectively
expanding along the structural dipole axis compared to the opposite
direction.

To express these slopes in dimensionless form we normalise them by an
approximate FRW comoving Hubble factor $H_{\rm com}(z)$ appropriate for the
TNG300 cosmology. Using a flat $\Lambda$CDM model consistent with the
simulation,
\begin{equation}
  H_{\rm phys}(z) = H_0\,\sqrt{\Omega_{\rm m}(1+z)^3 + \Omega_\Lambda},
  \qquad
  H_{\rm com}(z) = \frac{H_0}{h}\,
    \frac{\sqrt{\Omega_{\rm m}(1+z)^3 + \Omega_\Lambda}}{1+z},
\end{equation}
with $H_0 = 67.74\ {\rm km\,s^{-1}\,Mpc^{-1}}$,
$h = 0.6774$, $\Omega_{\rm m} = 0.3089$ and
$\Omega_\Lambda = 0.6911$. This matches our use of comoving radii in
Mpc$/h$, so the ratios $\delta H/H_{\rm com}$ and $\Delta H/H_{\rm com}$ are
dimensionless expansion anomalies.

Across the snapshot sequence $\{33,40,50,59,67,72,78,91\}$ we find a
remarkably coherent signal. The all–sky slope is consistently negative, with
\begin{equation}
  \frac{\delta H_{\rm all}}{H_{\rm com}}
  \simeq -(3.2\text{--}4.6)\times 10^{-3},
\end{equation}
indicating a mild net contraction relative to the FRW expectation, as
anticipated from the ``virialisation as viscosity'' picture. The hemispheric
fits are strongly asymmetric: in the hemisphere aligned with the structural
dipole the effective expansion rate is more negative, while in the opposite
hemisphere it is closer to zero or even mildly positive. The corresponding
directional contrast lies in the range
\begin{equation}
  \frac{\Delta H}{H_{\rm com}}
  \simeq -(3.5\text{--}6.4)\times 10^{-3},
\end{equation}
with the largest amplitudes appearing at $z \simeq 0.6$ and $z \simeq 0.2$,
where the structural dipole itself is strongest. The time evolution of
$\delta H_{\rm all}/H_{\rm com}$ and $\Delta H/H_{\rm com}$ is shown in
Figure~\ref{fig:expansion_dipole_time_series}.

Several points are worth noting. First, because we explicitly subtract the
mass–weighted bulk velocity vector before fitting the slopes, these
anisotropic expansion proxies are not kinematic artefacts of a drifting
simulation volume. The bulk–flow component emphasised in
Section~\ref{subsec:bulk_flow} (with $v_{\rm bulk}/c \sim 5\times 10^{-5}$
at late times) has been removed by construction. What remains is a combined
signal from large–scale infall, shear and any additional non–Markovian
response. Second, the directional contrast $\Delta H/H_{\rm com}$ is
coherent across $\sim 10$~Gyr of evolution and tracks the same preferred axis
as the structural dipole itself, reinforcing the interpretation of a single
long–wavelength mode modulating both the mass distribution and the local
expansion state.

Numerically, the amplitudes we obtain here are of order
$|\Delta H/H_{\rm com}| \sim 5\times 10^{-3}$, roughly two orders of
magnitude larger than the $\sim 10^{-5}$ expansion dipole predicted by the
measured memory kernel in our earlier work when evaluated at horizon scales.
This is expected: the present analysis is deliberately crude, using all
haloes at all radii in a single global fit, and it mixes virial motions,
filamentary shear and genuine Hubble–like flow into a single effective
parameter. In the next stage of the project we therefore refine this
measurement to approach the memory–kernel regime: we will (i) restrict the
fit to shells in $r$ where Hubble flow dominates, (ii) construct a
full–sky map of the expansion anomaly $\delta H(\hat{n})$ and its dipole
component, and (iii) propagate the measured structural dipole and bulk–flow
properties through the previously calibrated kernel to isolate the residual
non–Markovian contribution.

Nevertheless, this first measurement already demonstrates that the
large–scale structural dipole is accompanied by a coherent directional
expansion anomaly in TNG300 once trivial kinematic motion is removed. The
structural mode does not merely sit in the density field; it modulates the
effective expansion rate at the percent level across Gyr time–scales.
\begin{figure}
  \centering
  \includegraphics[width=0.7\textwidth]{figures/fig_expansion_dipole_time_series.png}
  \caption{Time evolution of the all–sky expansion anomaly
  $\delta H_{\rm all}/H_{\rm com}$ (circles) and the hemispheric expansion
  contrast $\Delta H/H_{\rm com}$ along the structural dipole axis (squares)
  in TNG300. The signal is coherent across $2 \gtrsim z \gtrsim 0.2$ and
  reaches $|\Delta H/H_{\rm com}|\sim 5\times 10^{-3}$ at late times.}
  \label{fig:expansion_dipole_time_series}
\end{figure}

\subsection{Directional expansion response in radial shells}
\label{sec:expansion_shells}

The hemispheric fits in Sec.~\ref{sec:expansion_dipole} treat the box as a single domain. To test whether the directional response of the velocity field is a genuine large–scale mode rather than a local feature near the box centre, we repeated the analysis in concentric comoving shells around the box centre. For each snapshot we split the volume into three radial shells,
\begin{equation}
20 \le r / (h^{-1}\mathrm{Mpc}) < 80, \quad
80 \le r / (h^{-1}\mathrm{Mpc}) < 140, \quad
140 \le r / (h^{-1}\mathrm{Mpc}) \le 205,
\end{equation}
and fit a dipole to the local Hubble residuals after subtracting the bulk flow and the fiducial FRW value.

As before, halo positions are measured relative to the box centre and velocities are peculiar velocities from TNG300. For each halo we define the radial unit vector $\hat{\mathbf{r}} = \mathbf{r}/|\mathbf{r}|$ and the radial peculiar velocity $v_r = \mathbf{v}\cdot\hat{\mathbf{r}}$. In each shell we then fit
\begin{equation}
\delta H(\hat{\mathbf{n}}) \equiv \frac{v_r}{r} - H_{\mathrm{FRW}}(z)
= a + \mathbf{b}\cdot\hat{\mathbf{n}},
\end{equation}
using a mass–weighted least–squares estimator over all halos in that shell.\footnote{%
TNG velocities are purely peculiar, with the background Hubble flow already subtracted. The large negative monopole $a/H_{\mathrm{FRW}}\simeq -1$ simply reflects this convention and is not a physical global contraction. In what follows we ignore $a$ and focus only on the directional structure $\mathbf{b}$.}
Before the fit we subtract the bulk flow vector $\mathbf{v}_{\rm bulk}(z)$ measured in Sec.~\ref{sec:bulk_flow}, so the resulting dipole $\mathbf{b}$ is by construction orthogonal to a pure kinematic dipole.

For each snapshot and shell we record the dimensionless amplitudes
\begin{equation}
\frac{|b|}{H_{\mathrm{FRW}}}, \qquad
\frac{b_{\parallel}}{H_{\mathrm{FRW}}}
= \frac{\mathbf{b}\cdot\hat{\mathbf{d}}_{\Sigma}}{H_{\mathrm{FRW}}}, \qquad
\frac{b_{\perp}}{H_{\mathrm{FRW}}}
= \sqrt{\frac{|b|^2 - b_{\parallel}^2}{H_{\mathrm{FRW}}^2}},
\end{equation}
where $\hat{\mathbf{d}}_{\Sigma}$ is the structural dipole direction measured from the mass distribution.

Across all snapshots, the monopole term sits at $a/H_{\mathrm{FRW}}\simeq -1$ in every shell, as expected from the peculiar–velocity convention, and does not carry physical information by itself. The dipole amplitudes, however, are stable and radial–scale dependent:
\begin{itemize}
  \item Inner shell ($20$–$80\,h^{-1}\mathrm{Mpc}$): we find $|b|/H_{\mathrm{FRW}} \simeq (2.0$–$2.6)\times 10^{-2}$ from $z\simeq 2$ to $z\simeq 0.2$, with a mix of positive and negative $b_{\parallel}$ depending on epoch. This is the most non–linear part of the box and shows the largest directional response.
  \item Middle shell ($80$–$140\,h^{-1}\mathrm{Mpc}$): here $|b|/H_{\mathrm{FRW}} \simeq (1.3$–$1.9)\times 10^{-2}$, with $b_{\parallel}/H_{\mathrm{FRW}}\simeq -(1.2$–$1.7)\times 10^{-2}$ at all snapshots. The dipole is already locked anti–parallel to the structural dipole.
  \item Outer shell ($140$–$205\,h^{-1}\mathrm{Mpc}$): this shell probes scales closest to the full box size. We find
  \begin{equation}
  \langle |b|/H_{\mathrm{FRW}} \rangle \simeq 1.1\times 10^{-2}, \qquad
  \langle b_{\parallel}/H_{\mathrm{FRW}} \rangle \simeq -9.6\times 10^{-3},
  \end{equation}
  with individual snapshots in the range $|b|/H_{\mathrm{FRW}} \simeq (0.8$–$1.2)\times 10^{-2}$ and
  $b_{\parallel}/H_{\mathrm{FRW}} \simeq -(0.7$–$1.1)\times 10^{-2}$. The perpendicular component stays at the few $\times 10^{-3}$ level. In other words, $80$–$90\%$ of the expansion dipole in the outer shell is aligned with the structural dipole direction.
\end{itemize}

Taken together with the hemispheric fits, these shell results show that:
\begin{enumerate}
  \item The effective expansion field develops a coherent dipole on scales up to $L\sim 200\,h^{-1}\mathrm{Mpc}$, with a fractional amplitude of order $10^{-2}$ in the peculiar–velocity frame.
  \item The dipole is not a local artefact near the box centre; it persists across all shells and remains strongly aligned with the structural dipole in the middle and outer shells.
  \item The outer shell, which is most relevant for comparison to the large–scale memory kernel, shows a remarkably stable anti–alignment of the expansion dipole with the structural dipole from $z\simeq 2$ down to $z\simeq 0.2$.
\end{enumerate}

In the rest of the paper we will treat the monopole shift as a gauge choice fixed by the peculiar–velocity convention and focus on the shell–averaged values of $b_{\parallel}/H_{\mathrm{FRW}}$ as the primary observable for constraining the late–time memory response on $\sim 200\,h^{-1}\mathrm{Mpc}$ scales.

\begin{figure*}
    \centering
    \includegraphics[width=0.48\textwidth]{figures/fig_expansion_shell_dipole_outer.png}%
    \hfill
    \includegraphics[width=0.48\textwidth]{figures/fig_expansion_shell_dipole_all.png}
    \caption{
    Shell–resolved expansion dipole measured from the radial velocity field after subtracting the box bulk flow.
    Left: dipole amplitude $|b|/H_{\rm FRW}$ (solid) and component aligned with the structural dipole $b_{\parallel}/H_{\rm FRW}$ (dashed) as a function of redshift in the outer shell, $140 < r < 205~h^{-1}\,{\rm Mpc}$.
    Right: same quantities for all three shells, $20$–$80$, $80$–$140$, and $140$–$205~h^{-1}\,{\rm Mpc}$. 
    In all shells the monopole shift stays close to $a/H_{\rm FRW}\simeq -1$, while the expansion dipole remains at the percent level, with a consistently negative aligned component $b_{\parallel}/H_{\rm FRW}<0$, indicating slower effective expansion along the structural dipole direction than perpendicular to it.}
    \label{fig:expansion_shell_dipole}
\end{figure*}
Figure~\ref{fig:expansion_shell_dipole} shows that the expansion dipole is shell–independent at the percent level and that the aligned component $b_{\parallel}/H_{\rm FRW}$ is robustly negative across radii, consistent with a coherent large–scale drag along the structural dipole.
The coherent bulk flow of the TNG300 box reaches $|v_{\rm bulk}|\simeq 15~{\rm km\,s^{-1}}$ at $z\simeq 0.2$, corresponding to a purely kinematic dipole of order $v_{\rm bulk}/c\sim 5\times 10^{-5}$, well below the observed CMB dipole amplitude ($\sim 10^{-3}$). Our expansion–dipole fits therefore probe a regime where kinematic anisotropy is small and can be cleanly subtracted, while the structural dipole and its memory–induced response remain the dominant anisotropic degrees of freedom in the volume.
\subsection{Radial shell tomography of the expansion dipole}
\label{sec:shell_tomography}

To localise the origin of the expansion dipole, I fit a linear model for the residual radial velocity field in three concentric shells around the box centre,
\begin{equation}
  v_r(\vec{r}) \;=\; a\, r \;+\; \vec{b}\!\cdot\!\hat{r},
\end{equation}
after subtracting the bulk flow measured in Section~\ref{sec:bulk_flow}. The monopole term $a$ captures a global shift in the effective Hubble rate, while the dipole vector $\vec{b}$ describes directional modulation of the expansion. I decompose $\vec{b}$ into a component along the structural dipole direction $\hat{d}_\Sigma$ and a perpendicular part,
\begin{equation}
  b_{\parallel} \equiv \vec{b}\!\cdot\!\hat{d}_\Sigma,
  \qquad
  b_{\perp} \equiv \big\|\vec{b} - b_{\parallel}\,\hat{d}_\Sigma\big\|.
\end{equation}
All amplitudes are quoted in units of the comoving background Hubble rate $H_{\rm FRW}(z)$.

The shells are defined in comoving radius as
\begin{align}
  \text{Shell 0}: &\quad 20 < r / (h^{-1}{\rm Mpc}) < 80, \\
  \text{Shell 1}: &\quad 80 < r / (h^{-1}{\rm Mpc}) < 140, \\
  \text{Shell 2}: &\quad 140 < r / (h^{-1}{\rm Mpc}) < 205.
\end{align}
Shell~0 is deeply non linear, Shell~1 is a transition region, and Shell~2 is the quasi linear regime closest to the box scale.

The outer shell carries the cleanest and most stable signal. Across all eight snapshots from $z\simeq2$ to $z\simeq0.2$ the dipole amplitude in Shell~2 remains at the one percent level,
\begin{equation}
  \big|b\big| / H_{\rm FRW} \;\simeq\; (1.0\text{--}1.2)\times10^{-2},
\end{equation}
with a consistently negative projection along the structural dipole,
\begin{equation}
  b_{\parallel} / H_{\rm FRW} \;\simeq\; -(0.7\text{--}1.1)\times10^{-2},
\end{equation}
and a smaller perpendicular piece
\begin{equation}
  b_{\perp} / H_{\rm FRW} \;\simeq\; (0.2\text{--}0.6)\times10^{-2}.
\end{equation}
In other words, the residual expansion is slowest along the direction of the large scale overdensity and fastest away from it, and this pattern is already present at $z\simeq2$ and survives down to $z\simeq0.2$.

The inner shells show larger raw amplitudes, $\lvert b\rvert/H_{\rm FRW}\sim0.015\text{--}0.025$, but these are dominated by non linear virial motions and local voids. The quasi linear, box scale behaviour is best traced by Shell~2. Figure~\ref{fig:expansion_shell_dipole} shows that the outer shell amplitude is almost time independent at the percent level, while $b_{\parallel}/H_{\rm FRW}$ becomes slightly more negative toward low redshift, in step with the late growth of the structural dipole $d_\Sigma(t)$.

Two robust facts emerge from this tomography:
\begin{itemize}
  \item The expansion dipole is a large scale phenomenon. It does not decay away when I exclude the inner $140~h^{-1}{\rm Mpc}$; instead it persists all the way out to $r\simeq205~h^{-1}{\rm Mpc}$.
  \item The dipole is structurally aligned. The parallel component $b_{\parallel}$ accounts for most of the amplitude in the outer shell, with the perpendicular piece subdominant. The anisotropic braking acts along the same axis picked out by the mass distribution.
\end{itemize}
This rules out the simple picture of a small local void or nearby attractor as the source of the effect. The anisotropic drag is spread across the entire volume and follows the filamentary spine of the cosmic web.
\subsection{Effective memory drag at 200\,$h^{-1}$Mpc}
\label{sec:drag_200mpc}

The shell analysis allows a first estimate of the effective integrated drag at the box scale. In the outer shell, the dimensionless expansion dipole sits at
\begin{equation}
  \left.\frac{\delta H}{H}\right|_{L\simeq200\,h^{-1}{\rm Mpc}}
  \;\equiv\;
  \frac{b_{\parallel}}{H_{\rm FRW}}
  \;\simeq\;
  -1.1\times10^{-2}.
\end{equation}
The corresponding structural dipole, convolved over the measured memory timescale, is of order
\begin{equation}
  d_\Sigma^{\rm eff}(L\simeq200\,h^{-1}{\rm Mpc})
  \;\sim\; 0.03,
\end{equation}
so that the effective drag parameter on these scales is
\begin{equation}
  \big|A\tau\big|_{\rm eff}(L\simeq200\,h^{-1}{\rm Mpc})
  \;\sim\;
  \frac{\lvert \delta H / H \rvert}{d_\Sigma^{\rm eff}}
  \;\simeq\; 0.3\text{--}0.4.
\end{equation}
For comparison, the small scale kernel measured in the viscosity paper gives
\begin{equation}
  \big|A\tau\big|(L\simeq50\,h^{-1}{\rm Mpc}) \;\simeq\; 1.1\times10^{-3},
\end{equation}
while the horizon scale fit to Planck and low redshift distance data implied
\begin{equation}
  \big|A\tau\big|(L\simeq{\rm horizon}) \;\simeq\; 3.4\times10^{-2}.
\end{equation}

Putting these three points together, the scale ladder for the integrated drag is no longer monotonic. The memory strength peaks at intermediate scales,
\begin{equation}
  \big|A\tau\big|(50\,h^{-1}{\rm Mpc}) \ll
  \big|A\tau\big|(200\,h^{-1}{\rm Mpc}) \gg
  \big|A\tau\big|({\rm horizon}),
\end{equation}
consistent with the qualitative picture where:
\begin{itemize}
  \item Small scales virialise quickly and relax, leaving a weak net drag.
  \item Intermediate scales host coherent flows and slow relaxation, producing the strongest backreaction.
  \item Horizon scales average over many uncorrelated modes, reducing the net anisotropic memory.
\end{itemize}
In this sense the TNG300 box behaves like the ``resonant band'' of the cosmic memory kernel: it sits at the scale where coherent structural history has the largest lever arm on the late time expansion.
\subsection{Tomographic rejection of a local--void explanation}
\label{sec:tomography_void}

A common counter argument to late time dipoles in the Hubble expansion is that they may be caused by sitting in an unusually large local void or near a single dominant attractor. The shell based analysis above allows a direct test of this picture.

If the anisotropy were mainly a local effect, one would expect the expansion dipole to weaken rapidly with radius once the problematic region is excluded. In contrast, the fitted dipole in the outer shell, $140<r/(h^{-1}{\rm Mpc})<205$, remains at the one percent level and is almost purely aligned with the structural axis,
\begin{equation}
  \left.\frac{\lvert b\rvert}{H_{\rm FRW}}\right|_{\rm outer}
  \simeq 1.1\times10^{-2},
  \qquad
  \frac{b_{\parallel}}{\lvert b\rvert} \gtrsim 0.9.
\end{equation}
The box expands more slowly toward the large scale overdensity and more rapidly toward the opposite side, even when I ignore the inner $140~h^{-1}{\rm Mpc}$ completely. The anisotropic braking is therefore not a small bubble we happen to sit in, but a global mode that threads the entire volume.

The innermost shell does show different behaviour at late times, with a positive $b_{\parallel}$ in some snapshots, consistent with local void dynamics and non linear flows. However, this local pattern is quickly overwhelmed by the negative, structure aligned signal in the intermediate and outer shells. The large scale drag wins once we look beyond $\sim100~h^{-1}{\rm Mpc}$.

In tomographic language, the expansion dipole is a box scale field line: it tracks the gradient of the gravitational potential across the simulation, not just the quirks of the immediate neighbourhood.

\section{Results}
\label{sec:results}

\subsection{Structural dipole and isotropy baseline}
\label{sec:results_structural}

I start by measuring the structural dipole of the subhalo distribution in TNG300 at $z\simeq0$, using snapshot $99$ and the full catalogue of $N\simeq1.45\times10^{7}$ subhalos. For each random orientation I define a unit vector $\hat{n}$, split the box into a near and far hemisphere, and compute a dipole estimator based on the mass weighted number counts. Repeating this for $2000$ random orientations gives a null distribution for the dipole amplitude under the hypothesis of statistical isotropy.

The resulting distribution has mean and scatter
\begin{equation}
  \mu\_{\rm null} \;=\; 0.0422,
  \qquad
  \sigma\_{\rm null} \;=\; 0.0216,
\end{equation}
with a $3\sigma$ upper tail at $\mu\_{\rm null}+3\sigma\_{\rm null}\simeq0.106$.
This defines the structural noise floor of a $205\,h^{-1}{\rm Mpc}$ box with $\Lambda$CDM initial conditions. Any single realisation with amplitude of a few percent is fully compatible with cosmic variance.

I then measure the structural dipole in the fixed simulation frame for eight snapshots from $z\simeq2$ to $z\simeq0.2$. The amplitudes are
\begin{equation}
  d\_\Sigma(t) \;\simeq\; \{\,0.030,\ 0.031,\ 0.037,\ 0.048,\ 0.033,\ 0.074,\ 0.064,\ 0.061\,\},
\end{equation}
for snapshots $(33,40,50,59,67,72,78,91)$ respectively. From $z\simeq2$ down to $z\simeq0.8$ the values cluster near the null mean. At $z\simeq0.5$ the box develops a strong structural mode with $d\_\Sigma\simeq0.074$, which stays above $0.06$ down to $z\simeq0.2$, corresponding to a $3.4\sigma$ deviation from the null.

A direction cosine matrix built from the eight structural dipole vectors shows that the direction is coherent across time. The cosine of the angle between the earliest and latest vectors is $\cos\theta\simeq0.87$, and most off diagonal entries lie above $0.8$. The large scale structural mode that emerges around $z\simeq0.5$ is therefore not a transient, but a stable orientation of the cosmic web across about $8$ Gyr of evolution.

Raising the subhalo mass threshold from no cut to $M>10^{10}$, $10^{11}$ and $10^{12}\,h^{-1}M\_\odot$ at $z\simeq0.5$ increases the structural dipole amplitude from $d\_\Sigma\simeq0.074$ to $0.082$, $0.091$ and $0.120$, while keeping the direction almost unchanged. Massive, biased tracers sit deeper in the same mode. The dipole is therefore a genuine feature of the large scale structure, not a sampling accident in the low mass population.

\subsection{Bulk flow and kinematic control}
\label{sec:results_bulk}

To rule out a simple kinematic interpretation I compute the mass weighted bulk flow of the same subhalo population at each snapshot. The net velocity of the box never exceeds $15.5\ {\rm km\,s^{-1}}$, even at late times, and is much smaller than typical observed bulk flows on similar scales. The TNG300 volume is effectively at rest in its own CMB frame.

More importantly, the alignment between the bulk flow vector and the structural dipole is not stable. At $z\simeq0.5$ the angle is about $127^\circ$, consistent with a temporary infall episode into a forming overdensity. At $z\simeq0.2$ the angle is about $93^\circ$, so the bulk motion is nearly orthogonal to the structural axis once the large scale mode has settled. There is no long lived state where the bulk flow simply points at, or away from, the structural dipole.

Given that the formal kinematic dipole from such small bulk motions would be at the level $v\_{\rm bulk}/c\sim5\times10^{-5}$, two orders of magnitude below the expansion signals measured below, I treat the bulk flow as a nuisance that can be subtracted snapshot by snapshot before fitting the expansion field.

\subsection{Expansion dipole in hemispheres}
\label{sec:results_hemi}

With the structural axis and bulk motion under control, I turn to the expansion field. For each snapshot I subtract the bulk flow from the subhalo velocities, place an observer at the box centre, and fit a linear relation
\begin{equation}
  v\_r \;=\; H\_{\rm eff}(z)\,r \;+\; \delta v\_r,
\end{equation}
in three samples: the full sky, the hemisphere that points along the structural dipole, and the opposite hemisphere. I quote the residual shifts in the effective Hubble rate relative to the comoving background $H\_{\rm FRW}(z)$.

The full sky residual is always negative and of order a few parts in a thousand,
\begin{equation}
  \left.\frac{\delta H}{H}\right|\_{\rm all}
  \;\simeq\; -(3.2\text{--}4.6)\times10^{-3}
\end{equation}
for $2\gtrsim z\gtrsim0.2$, consistent with the net drag of virial motions in the volume.

The hemispheric contrast is larger. The hemisphere that looks into the structural overdensity consistently expands more slowly, while the opposite hemisphere expands slightly faster than the background. The inferred dipole in the expansion rate,
\begin{equation}
  \left.\frac{\delta H}{H}\right|\_{\rm dip}
  \;\equiv\; \frac{1}{2H}\,\big(H\_{\rm north} - H\_{\rm south}\big),
\end{equation}
sits in the range
\begin{equation}
  \left.\frac{\delta H}{H}\right|\_{\rm dip}
  \;\simeq\; -(3.5\text{--}6.4)\times10^{-3},
\end{equation}
with the largest values at $z\simeq0.6$ and $z\simeq0.2$. This already shows a clear anisotropic braking along the structural axis, but the hemisphere split still mixes contributions from small and large scales. To isolate the cosmological part I add a radial decomposition.

\subsection{Radial shell tomography}
\label{sec:results_shell}

I now split the box into three spherical shells,
\begin{align}
  20 < r / (h^{-1}{\rm Mpc}) &< 80, \\
  80 < r / (h^{-1}{\rm Mpc}) &< 140, \\
  140 < r / (h^{-1}{\rm Mpc}) &< 205,
\end{align}
and fit a full dipole model to the bulk flow subtracted radial velocities in each shell,
\begin{equation}
  v\_r(\vec{r}) \;=\; a\,r \;+\; \vec{b}\!\cdot\!\hat{r}.
\end{equation}
Here $a$ is a monopole shift in the Hubble rate and $\vec{b}$ is a three dimensional dipole vector. I decompose $\vec{b}$ into components parallel and perpendicular to the structural dipole direction.

The inner shells show large amplitudes, with $\lvert b\rvert/H\_{\rm FRW}$ in the range $0.015$ to $0.025$, but are clearly dominated by non linear flows and local voids. The signs of $b\_{\parallel}$ flip with time in Shell 0, and inspection of the velocity field suggests that local structure near the observer plays a large role there.

The outer shell, $140<r/(h^{-1}{\rm Mpc})<205$, behaves very differently. Across all eight snapshots the fitted dipole amplitude and its projection along the structural axis remain remarkably stable,
\begin{equation}
  \left.\frac{\lvert b\rvert}{H\_{\rm FRW}}\right|\_{\rm outer}
  \;\simeq\; (1.0\text{--}1.2)\times10^{-2},
\end{equation}
\begin{equation}
  \left.\frac{b\_{\parallel}}{H\_{\rm FRW}}\right|\_{\rm outer}
  \;\simeq\; -(0.7\text{--}1.1)\times10^{-2},
\end{equation}
with a smaller perpendicular part $b\_{\perp}/H\_{\rm FRW}\simeq(0.2\text{--}0.6)\times10^{-2}$. The sign of $b\_{\parallel}$ is always negative: the expansion is slower along the structural dipole and faster in the opposite direction. The amplitude in the outer shell does not decay when I move the inner boundary from $80$ to $140\,h^{-1}{\rm Mpc}$. The anisotropic braking is not a local bubble. It is a box scale mode that persists all the way out to the edge of the volume.

At $z\simeq0.2$ the outer shell fit gives
\begin{equation}
  \left.\frac{\lvert b\rvert}{H\_{\rm FRW}}\right|\_{z\simeq0.2}
  \;\simeq\; 1.08\times10^{-2},
  \qquad
  \left.\frac{b\_{\parallel}}{H\_{\rm FRW}}\right|\_{z\simeq0.2}
  \;\simeq\; -1.02\times10^{-2},
\end{equation}
so more than $90\%$ of the dipole amplitude lies along the structural axis at the present epoch.

\subsection{Effective drag at the box scale}
\label{sec:results_drag}

The outer shell behaviour can be summarised as a single number. If I denote by $d^{\rm eff}\_\Sigma(L)$ the effective structural dipole that sources the expansion dipole on scale $L\simeq200\,h^{-1}{\rm Mpc}$, then the observed relation
\begin{equation}
  \left.\frac{\delta H}{H}\right|\_{L\simeq200\,h^{-1}{\rm Mpc}}
  \;\simeq\; -1.1\times10^{-2},
  \qquad
  d^{\rm eff}\_\Sigma \;\sim\; 0.03,
\end{equation}
implies an integrated memory drag of order
\begin{equation}
  \big|A\tau\big|\_{\rm eff}(L\simeq200\,h^{-1}{\rm Mpc})
  \;\sim\; 0.3\text{--}0.4.
\end{equation}

For comparison, the domain based kernel measurement at $L\simeq50\,h^{-1}{\rm Mpc}$ in the viscosity analysis gave $\lvert A\tau\rvert\simeq1.1\times10^{-3}$, while the horizon scale fit to Planck and low redshift distances gave $\lvert A\tau\rvert\simeq3.4\times10^{-2}$. The TNG300 anisotropy experiment therefore adds a new point to the scale ladder, and this new point sits at the top. The effective memory drag appears to peak at intermediate, box sized scales where coherent flows have the longest time to react to accumulated structural history.


\section{Conclusions}
\label{sec:conclusions}

The aim of this logbook was simple. I wanted to know whether a finite, structured memory of non linear evolution can turn uneven structure formation into a measurable dipole in the late time expansion, without breaking the statistical isotropy of the initial conditions. The TNG300 experiments above give a clear answer.

First, there is a real structural dipole in the simulation volume. It emerges around $z\simeq0.5$, reaches an amplitude $d\_\Sigma\simeq0.07$, and keeps a stable direction within about $30^\circ$ down to $z\simeq0.2$. A Monte Carlo isotropy test shows that this is a genuine large scale mode and not a $3\sigma$ fluke.

Second, the box does not simply drift in that direction. The net bulk flow never exceeds $16\ {\rm km\,s^{-1}}$, and its direction is not locked to the structural axis. At late times the flow is almost orthogonal to the structural dipole. The small kinematic dipole has been measured and subtracted. What is left in the radial velocity field is not a moving box artefact.

Third, once bulk motion is removed, the effective Hubble rate is not the same in all directions inside the box. Hemispheric fits along and opposite to the structural dipole show a hemispheric expansion contrast at the level of a few parts in a thousand, with the overdense side always expanding more slowly. A full dipole fit in concentric shells then pins down where the cleanest signal lives. The inner $80\,h^{-1}{\rm Mpc}$ are messy, as expected. The outer shell, between $140$ and $205\,h^{-1}{\rm Mpc}$, carries a neat, quasi linear signal: a one percent expansion dipole that is almost perfectly anti aligned with the structural dipole and stable from $z\simeq2$ to $z\simeq0.2$.

Fourth, this shell tomography kills the easy escape route. The anisotropic braking is not a local bubble, not a single lucky void, and not a short range artefact that fades away once I look far enough out. It is a box scale mode that threads the full volume. The universe in the simulation expands a bit slower along the filamentary spine of the cosmic web and a bit faster toward the underdense side, and this pattern extends all the way to the box edge.

Finally, when I fold this back into the memory framework of the companion papers, the picture is quite sharp. The small scale kernel on $50\,h^{-1}{\rm Mpc}$ domains is weak. The horizon scale kernel from CMB and distance fits is mild. The TNG300 anisotropy experiment shows that the effective drag peaks at intermediate, $200\,h^{-1}{\rm Mpc}$ scales, with $\lvert A\tau\rvert\_{\rm eff}\sim0.3\text{--}0.4$. The box scale behaves like the resonant band of the cosmic memory kernel, where accumulated structural history has the strongest lever arm on the expansion.

From a conservative point of view, this work provides a benchmark. In a standard $\Lambda$CDM hydrodynamic simulation with realistic structure, a percent level expansion dipole on two hundred megaparsec scales is not an exotic signal. It is the natural outcome of living in a finite, uneven piece of the universe. Any attempt to use real dipole measurements to test non Markovian cosmology has to clear at least this bar.

From the more ambitious point of view, this closes a loop. The viscosity paper measured the kernel on small scales. The scale dependent study showed how the memory time grows with scale. The present logbook shows what happens when all that history points in a preferred direction. Uneven structure does feed into anisotropic expansion, and the effect is largest right where one would expect it from a universe that remembers.


\subsection{TNG50 structural dipole cross--check}
\label{sec:tng50_dipole}

To test whether the structural dipole is a peculiarity of the TNG300 volume or a generic feature of late--time structure, we repeated the analysis in the higher--resolution but much smaller TNG50--1 box. The TNG50--1 volume has side length $L_{\rm box} = 35\,{\rm Mpc}/h$, so one expects larger fractional fluctuations from cosmic variance than in the $L_{\rm box} = 205\,{\rm Mpc}/h$ TNG300 box.

We analysed seven snapshots,
\[
{\rm SNAP} = \{40, 50, 59, 67, 72, 78, 91\},
\]
using all subhaloes with well--defined positions and masses. The group catalogues contain some files without usable subhalo position fields (no \texttt{SubhaloPos} or \texttt{SubhaloCM}); these chunks were skipped. For the latest snapshot, ${\rm SNAP}=91$, this leaves $N_{\rm file}=501$ usable groupcat files out of the full set, and a total of $N_{\rm sub} \simeq 5.8\times 10^6$ subhaloes.

As in the TNG300 analysis, we define the structural dipole vector by
\begin{equation}
 \boldsymbol{d}_\Sigma \;\equiv\;
 \frac{\sum_i m_i \,\hat{\boldsymbol{n}}_i}{\sum_i m_i}\,,
 \qquad
 \hat{\boldsymbol{n}}_i \;=\;
 \frac{\boldsymbol{x}_i - \boldsymbol{x}_{\rm c}}{\lvert \boldsymbol{x}_i - \boldsymbol{x}_{\rm c} \rvert}\,,
\end{equation}
with the box centre at $\boldsymbol{x}_{\rm c} = (L_{\rm box}/2,L_{\rm box}/2,L_{\rm box}/2)$ and $m_i$ the subhalo masses in native TNG units. The dipole amplitude is $d_\Sigma = \lvert \boldsymbol{d}_\Sigma \rvert$.

For ${\rm SNAP}=91$ we find
\begin{equation}
 d_\Sigma^{\rm (TNG50)}(z \simeq 0.2)
 \;\simeq\; 0.075\,.
\end{equation}
This is a factor of a few larger than the typical TNG300 amplitudes on $L_{\rm box}=205\,{\rm Mpc}/h$ (Section~\ref{sec:structural_dipole_timeseries}), exactly in line with the expectation that smaller volumes exhibit stronger fractional anisotropies.

The most important diagnostic is the coherence of the dipole direction over cosmic time. Defining unit vectors
\begin{equation}
 \hat{\boldsymbol{d}}_\Sigma({\rm SNAP}) \;=\;
 \frac{\boldsymbol{d}_\Sigma({\rm SNAP})}{\lvert \boldsymbol{d}_\Sigma({\rm SNAP}) \rvert}\,,
\end{equation}
we construct the direction cosine matrix
\begin{equation}
 C_{ij} \;=\;
 \hat{\boldsymbol{d}}_\Sigma({\rm SNAP}_i)\,\cdot\,
 \hat{\boldsymbol{d}}_\Sigma({\rm SNAP}_j)\,,
\end{equation}
for the seven snapshots listed above. Numerically we obtain
\begin{equation}
 C_{ij} \;\approx\;
 \begin{pmatrix}
 1.00 & 0.98 & 0.999 & 0.992 & 0.985 & 0.995 & 0.987 \\
 0.98 & 1.00 & 0.989 & 0.995 & 0.990 & 0.975 & 0.947 \\
 0.999 & 0.989 & 1.00 & 0.997 & 0.992 & 0.995 & 0.983 \\
 0.992 & 0.995 & 0.997 & 1.00 & 0.998 & 0.992 & 0.974 \\
 0.985 & 0.990 & 0.992 & 0.998 & 1.00 & 0.991 & 0.972 \\
 0.995 & 0.975 & 0.995 & 0.992 & 0.991 & 1.00 & 0.994 \\
 0.987 & 0.947 & 0.983 & 0.974 & 0.972 & 0.994 & 1.00
 \end{pmatrix}\!,
\end{equation}
with all off--diagonal entries $C_{ij} \gtrsim 0.95$.

Thus, in the TNG50 box the structural dipole direction is essentially locked in place from $z\simeq 1.5$ to $z\simeq 0.2$. The small volume is dominated by a single large--scale mode that imprints a fixed anisotropy on the mass distribution. This behaviour mirrors the strong directional coherence seen in TNG300, but at higher amplitude, as expected from the smaller survey volume.

Taken together, TNG50 and TNG300 show that the structural dipole is not an artefact of a particular box size or resolution. Different volumes, different resolutions, same basic story: one preferred structural axis that survives for several gigayears.

\subsection{Cross–box consistency: TNG50-1 as a small–volume stress test}
\label{sec:tng50_crosscheck}

To test whether the expansion dipole is an accident of the TNG300-1 volume, 
I repeated the full structural and kinematic analysis in the much smaller 
TNG50-1 box. The TNG50-1 volume has $L_{\rm box} = 35~{\rm Mpc}/h$, so all 
radii beyond $r \gtrsim 10~{\rm Mpc}/h$ are already deep in the non–linear 
regime. This makes TNG50-1 unsuitable as a precision amplitude calibrator, 
but it is an ideal stress test: if the signal were a fragile numerical 
artifact, it should fail here first.

Using the same centre–of–box definition and subhalo catalogue, I measured 
the structural dipole time series at the snapshots
\[
  \{40, 50, 59, 67, 72, 78, 91\}
  \quad \Rightarrow \quad
  z \simeq \{1.5, 1.0, 0.8, 0.6, 0.5, 0.4, 0.2\}.
\]
The dimensionless structural dipole amplitude is large,
$d_\Sigma \simeq 0.88$ at all times, and the directional 
coherence matrix is essentially unity,
\begin{equation}
  C_{ij} \equiv \hat{\boldsymbol{d}}_\Sigma(z_i)\cdot 
  \hat{\boldsymbol{d}}_\Sigma(z_j) \simeq 1,
\end{equation}
indicating that the entire small box is dominated by a single 
filamentary or wall–like configuration that barely reorients over 
$\sim 7~{\rm Gyr}$ of evolution.

I then computed the bulk flow in the same way as for TNG300-1, using the 
mass–weighted mean velocity of the subhalo sample after a conservative 
mass cut $\tilde{M}_{\rm sub} > 1$ ($>10^{10}~{\rm M}_\odot/h$). The 
bulk–flow amplitudes remain small at all redshifts; for example,
at $z \simeq 0.2$ (snapshot 91) I find
\begin{equation}
  |\boldsymbol{v}_{\rm bulk}| \simeq 2.2~{\rm km\,s^{-1}}, \qquad
  \cos\theta\bigl(\boldsymbol{d}_\Sigma,\boldsymbol{v}_{\rm bulk}\bigr)
  \simeq -0.28,
\end{equation}
so the TNG50-1 volume is also close to the CMB rest frame and the 
structural axis is not trivially aligned with the residual motion.

After subtracting the bulk flow, I repeated the hemispheric expansion 
fits along the structural dipole axis, exactly as in the TNG300-1 analysis. 
For each snapshot I fit an effective expansion rate $H_{\rm eff}$ in the 
full sample, and in the ``north'' and ``south'' hemispheres defined by the 
sign of 
$\hat{\boldsymbol{r}}\cdot\hat{\boldsymbol{d}}_\Sigma$.
The hemispheric dipole estimator,
\begin{equation}
  \delta H_{\rm dip}(z) \equiv 
  \frac{H_{\rm eff}^{\rm (north)} - H_{\rm eff}^{\rm (south)}}{2},
\end{equation}
is then expressed relative to the comoving FRW value $H_{\rm FRW}(z)$.

Because TNG50-1 is so small and so non–linear, the monopole shifts
$H_{\rm eff} - H_{\rm FRW}$ are large: the global expansion rate is 
suppressed by of order unity, with $\delta H_{\rm all}/H \simeq -1.0$ 
for all redshifts considered. In spite of this, the hemispheric 
\emph{difference} is remarkably stable in fractional terms. The measured 
dipole components are
\begin{equation}
  \left.\frac{\delta H_{\rm dip}}{H}\right|_{\rm TNG50}
  \simeq
  -\,(0.7\text{--}1.4)\times 10^{-2},
\end{equation}
with the following representative values:
\begin{center}
\begin{tabular}{cccc}
\hline
Snapshot & $z$ & $\delta H_{\rm all}/H$ & $\delta H_{\rm dip}/H$ \\
\hline
40 & $1.5$ & $-1.006$ & $-6.8\times 10^{-3}$ \\
50 & $1.0$ & $-1.008$ & $-1.1\times 10^{-2}$ \\
59 & $0.8$ & $-1.009$ & $-1.2\times 10^{-2}$ \\
67 & $0.6$ & $-1.010$ & $-1.3\times 10^{-2}$ \\
72 & $0.5$ & $-1.009$ & $-1.4\times 10^{-2}$ \\
78 & $0.4$ & $-1.009$ & $-1.2\times 10^{-2}$ \\
91 & $0.2$ & $-1.011$ & $-1.4\times 10^{-2}$ \\
\hline
\end{tabular}
\end{center}
In other words, even in a tiny, strongly collapsed box whose global 
expansion is almost fully quenched, the directional \emph{anisotropy} 
of the effective Hubble rate sits naturally at the $\sim 1\%$ level 
and tracks the structural dipole axis.

I do not use TNG50-1 to recalibrate the amplitude of the scale–dependent 
memory kernel, because the box is too small to cleanly separate quasi–linear 
modes from virialised flows. However, this cross–check shows that the 
$\sim 1\%$ expansion dipole seen in TNG300-1 is not a fragile feature of 
one particular volume: a completely different box, with a much smaller 
size and a more extreme dynamical state, develops a comparable fractional 
anisotropy aligned with its dominant structural mode. The directional 
braking of expansion along the cosmic web spine is therefore a robust 
outcome of $\Lambda$CDM structure formation, not a numerical accident 
of a single simulation.
\subsubsection*{9. Cross–box consistency: TNG300 versus TNG50}

To test whether the hemispheric expansion dipole in TNG300 is a one–box fluke, I repeated the full pipeline on the higher–resolution TNG50 volume using the same structural axis estimator and the same least–squares expansion fit.

For TNG300, the hemispheric fit along the structural dipole direction gives
\begin{align}
\frac{\Delta H_{\rm all}}{H_{\rm FRW}} &\simeq -(3.2\text{--}4.6)\times 10^{-3} \\
\frac{\Delta H_{\rm dip}}{H_{\rm FRW}} &\simeq -(3.5\text{--}6.0)\times 10^{-3}
\end{align}
for snapshots $33 \leq {\rm snap} \leq 91$ ($2 \gtrsim z \gtrsim 0.2$), with the sign indicating slower expansion in the overdense hemisphere. This is the result used in the main TNG300 analysis.

Running the identical estimator on TNG50, with structural dipole vectors taken from \texttt{tng50\_dipole\_time\_series.py}, yields
\begin{align}
\frac{\Delta H_{\rm all}}{H_{\rm FRW}} &\simeq -1.01 \\
\frac{\Delta H_{\rm dip}}{H_{\rm FRW}} &\simeq -(0.7\text{--}1.4)\times 10^{-2}
\end{align}
for the overlapping redshift range $1.5 \geq z \geq 0.2$ (snapshots 40, 50, 59, 67, 72, 78, 91 in TNG50). The monopole term at the TNG50 box scale is essentially $\Delta H_{\rm all}/H \simeq -1$, that is, the effective expansion rate inside the $35\,{\rm Mpc}/h$ box is almost fully cancelled by non–linear motions. This is expected for such a small, deeply clustered volume and simply confirms that TNG50 is not a fair sample of the global Hubble flow.

The key point is the \emph{dipole} amplitude. Even in this strongly non–linear environment, the structural axis carries a coherent hemispheric expansion asymmetry at the per cent level,
\[
\left|\frac{\Delta H_{\rm dip}}{H_{\rm FRW}}\right|_{\rm TNG50}
\simeq (0.7\text{--}1.4)\%
\]
comparable to, and in fact somewhat larger than, the few times $10^{-3}$ dipole detected in TNG300 at $L\simeq 200\,{\rm Mpc}/h$.

This cross–box test shows:

\begin{itemize}
  \item The estimator itself is robust: applied to a completely independent volume with different resolution and different environment, it again picks up a percent–level expansion dipole aligned with the structural axis.
  \item The difference in monopole level (\mbox{$\Delta H/H\simeq -10^{-3}$} in TNG300, \mbox{$\Delta H/H\simeq -1$} in TNG50) is simply the difference between a quasi–linear $205\,{\rm Mpc}/h$ box and a non–linear $35\,{\rm Mpc}/h$ patch.
  \item The signal is therefore not an artefact of TNG300 alone, but a generic feature of structure–induced anisotropy that persists across box sizes and resolutions.
\end{itemize}

\begin{figure}[t]
  \centering
  \includegraphics[width=0.48\textwidth]{figures/fig_compare_expansion_dipole_TNG300_TNG50.png}
  \caption{
  Comparison of the hemispheric expansion fit between TNG300 and TNG50.
  Left panel: monopole shift $\Delta H_{\rm all}/H_{\rm FRW}$ as a function of redshift. TNG300 (solid) sits at the $10^{-3}$ level, while TNG50 (dashed) is effectively fully decelerated at its box scale, as expected in a small, non–linear region.
  Right panel: dipole term $\Delta H_{\rm dip}/H_{\rm FRW}$ measured along the structural axis. Both volumes show a coherent per cent–level expansion dipole, confirming that the effect is not unique to TNG300 and that the estimator is robust across resolutions.
  }
  \label{fig:compare_expansion_dipole_TNG300_TNG50}
\end{figure}
\subsubsection*{10. TNG50 shell tomography as a resolution check}

For TNG300, the radial shell decomposition showed that the cleanest cosmological signal lives in the outer shell $r\in[140,205]\,{\rm Mpc}/h$, where the expansion dipole amplitude is $\lvert b\rvert/H \simeq 0.011$ and the component along the structural axis satisfies $b_{\parallel}/H \simeq -0.010$, that is, a $\sim 1.1\%$ braking of the expansion along the overdense direction.

I repeated the same shell fitting on TNG50 at its own box scale, using
\[
r \in [5,15]\ {\rm Mpc}/h,\quad
[15,25]\ {\rm Mpc}/h,\quad
[25,30]\ {\rm Mpc}/h
\]
with bulk flow subtraction and the same linear model for the residual velocities in each shell.

In the outer shell of TNG50 ($r\in[25,30]\,{\rm Mpc}/h$), the fitted dipole amplitude and its projection along the structural axis are
\begin{align}
\left|\frac{b}{H_{\rm FRW}}\right|_{\rm outer}
&\simeq (2.1\text{--}4.5)\times 10^{-2}\,, \\
\frac{b_{\parallel}}{H_{\rm FRW}}
&\simeq (1.6\text{--}4.2)\times 10^{-2}\,,
\end{align}
for $1.5 \geq z \geq 0.2$, with shell occupancies $N_{\rm shell}\sim 200$ haloes per snapshot. In other words, at the edge of the $35\,{\rm Mpc}/h$ TNG50 box the residual expansion field carries a few per cent dipole aligned with the structural axis.

Two points are important here:

\begin{itemize}
  \item The sign of $b_{\parallel}$ in the TNG50 outer shell is positive, that is, expansion is faster along the structural axis rather than slower. This is not a contradiction with the TNG300 result. It simply reflects the fact that TNG50 is a small, strongly non–linear patch where the structural axis is dominated by a local void and nearby attractor, not a quasi–linear mode on $200\,{\rm Mpc}/h$ scales. The estimator is faithfully reporting the local geometry.
  \item The amplitude, at the few per cent level, is again large compared to the FRW baseline and completely consistent with the idea that structural history imprints a strong directional component into the effective expansion rate, even in a small, fully non–linear box.
\end{itemize}

In the main analysis I therefore treat the TNG300 outer shell as the cosmological measurement at $L\simeq 200\,{\rm Mpc}/h$, and TNG50 as a resolution and methodology check. The fact that the same pipeline finds a strong, structure–aligned expansion dipole in both boxes, despite very different volumes and environments, significantly weakens any claim that the TNG300 result is a numerical accident.

\begin{figure}[t]
  \centering
  \includegraphics[width=0.48\textwidth]{figures/tng50_fig_expansion_shell_dipole_outer.png}
  \caption{
  TNG50 expansion dipole from the radial shell fit.
  The outer shell $r\in[25,30]\,{\rm Mpc}/h$ carries a robust few per cent dipole amplitude $|b|/H$ whose projection along the structural axis, $b_{\parallel}/H$, remains of order $(1.5\text{--}4)\times 10^{-2}$ from $z=1.5$ to $z=0.2$. This confirms that the shell–based estimator detects strong, structure–aligned anisotropy even in a small, non–linear volume and supports the use of the same method in the much larger TNG300 box for cosmological inference.
  }
  \label{fig:tng50_shell_dipole_outer}
\end{figure}

\subsubsection{SDSS DR8 structural dipole (angular only)}
\label{sec:sdss_structural_dipole}

As a first observational sanity check, we constructed a purely angular structural dipole from the SDSS DR8 galaxy sample used in the $\hat{\phi}$ analysis. We use the catalogue
\texttt{sdss\_dr8\_analysis\_base\_v1.csv} and apply the following selection:
\begin{itemize}
  \item Reliability cut: \texttt{RELIABLE} $=1$.
  \item Redshift range: $0.02 \le z \le 0.20$.
  \item Sky position from the \texttt{RA}, \texttt{DEC} columns (ICRS, degrees).
  \item Stellar-mass weighting using \texttt{LGM\_TOT\_P50}: $w \propto 10^{\mathrm{LGM\_TOT\_P50}}$.
\end{itemize}
After the reliability and redshift cuts, the working sample contains $N = 200{,}000$ galaxies. We convert $(\mathrm{RA},\mathrm{DEC})$ to unit vectors $\hat{\mathbf{n}}_i$ on the celestial sphere and define the mass-weighted structural dipole as
\begin{equation}
  \mathbf{d}_\Sigma
  \;=\;
  \frac{\sum_i w_i \,\hat{\mathbf{n}}_i}{\sum_i w_i}, 
  \qquad
  d_\Sigma \equiv \left\lVert \mathbf{d}_\Sigma \right\rVert,
\end{equation}
with the corresponding right ascension and declination extracted from $\mathbf{d}_\Sigma$.

We first compute the dipole in six redshift bins between $z=0.02$ and $z=0.20$. In practice, the catalogue is populated only up to $z\simeq 0.11$ in this subset, so the three highest bins are empty. The three populated bins contain
$N = 53{,}406$, $85{,}822$ and $60{,}772$ galaxies, respectively, and yield
\begin{align}
  d_\Sigma(z \in [0.02,0.05)) &\simeq 0.63, 
  & (\mathrm{RA},\mathrm{DEC}) &\simeq (190^\circ, 39^\circ),\\
  d_\Sigma(z \in [0.05,0.08)) &\simeq 0.63, 
  & (\mathrm{RA},\mathrm{DEC}) &\simeq (189^\circ, 38^\circ),\\
  d_\Sigma(z \in [0.08,0.11)) &\simeq 0.63, 
  & (\mathrm{RA},\mathrm{DEC}) &\simeq (190^\circ, 34^\circ).
\end{align}
Over the full redshift range $0.02 \le z \le 0.11$ the global dipole is
\begin{equation}
  d_\Sigma^{\rm (global)} \simeq 0.63,
  \qquad
  (\mathrm{RA},\mathrm{DEC}) \simeq (189.5^\circ, 36.3^\circ).
\end{equation}

The large amplitude and the near-constancy of the dipole across redshift bins indicate that this is not a cosmological signal but rather a manifestation of the highly anisotropic SDSS DR8 sky footprint (dominated by the North Galactic Cap). In other words, the SDSS mass dipole is strongly geometry-driven: the vector $\mathbf{d}_\Sigma$ essentially points towards the centre of the survey window, and the amplitude $d_\Sigma \simeq 0.6$ primarily reflects the angular mask rather than an intrinsic anisotropy of the galaxy distribution.

This run therefore serves as a calibration of the survey-mask dipole. The next step is to construct an SDSS null distribution that preserves the angular selection function---for example by randomising the galaxy positions within the mask or by using a dedicated random catalogue with the same window and redshift selection---and to compare the measured $d_\Sigma$ to this null. Only after subtracting or marginalising over the mask-induced dipole can we meaningfully compare SDSS structural directions to the simulation-based structural dipoles measured in TNG300 and TNG50.

\subsubsection{SDSS DR8 structural dipole and isotropic-sky null test}
\label{sec:sdss_sdss_structural_dipole_null}

As a sanity check on the structural dipole estimator, I repeated the analysis on an SDSS~DR8 galaxy sample, using the same logic as in the TNG300 runs but now in full observational coordinates.

\paragraph{Sample definition.}
I start from the SDSS~DR8 analysis catalog
\texttt{sdss\_dr8\_analysis\_base\_v1.csv} with
$N_{\rm raw}=200{,}000$ entries. I apply:
\begin{itemize}
  \item a reliability cut ${\tt RELIABLE}=1$ (leaving $N=200{,}000$),
  \item removal of NaNs in RA, DEC, redshift $z$ and stellar mass proxy ${\tt LGM\_TOT\_P50}$,
  \item a redshift window $0.02 \le z \le 0.20$.
\end{itemize}
After these cuts the working sample still contains $N=200{,}000$ galaxies.
Each galaxy is assigned a weight
\begin{equation}
  w_i \propto 10^{\texttt{LGM\_TOT\_P50}}\;,
\end{equation}
and the sky position is converted from $(\mathrm{RA},\mathrm{DEC})$ to a unit vector
$\hat{\mathbf{n}}_i$.

\paragraph{Observed structural dipole.}
The mass-weighted structural dipole is defined as
\begin{equation}
  \mathbf{d}_\Sigma
    = \frac{\sum_i w_i \hat{\mathbf{n}}_i}{\sum_i w_i}\,,
  \qquad
  d_\Sigma = \bigl\lVert \mathbf{d}_\Sigma \bigr\rVert.
\end{equation}
For the full sample I find
\begin{align}
  d_{\Sigma,{\rm obs}}^{\rm global} & = 0.628 \,,\\
  (\mathrm{RA},\mathrm{DEC})_{\rm obs}^{\rm global}
    & \simeq (189.5^\circ,\,21.8^\circ)\,.
\end{align}
I also split the sample into six redshift bins
with edges at $z = 0.02, 0.05, 0.08, 0.11, 0.14, 0.17, 0.20$.
The first three bins are populated, the higher three are effectively empty.
For the populated bins I obtain:
\begin{itemize}
  \item Bin~0, $0.02 \le z < 0.05$:
        $N=53{,}406$,
        $d_{\Sigma,{\rm obs}} = 0.6275$,
        $(\mathrm{RA},\mathrm{DEC}) \simeq (189.9^\circ,23.2^\circ)$;
  \item Bin~1, $0.05 \le z < 0.08$:
        $N=85{,}822$,
        $d_{\Sigma,{\rm obs}} = 0.6316$,
        $(\mathrm{RA},\mathrm{DEC}) \simeq (189.3^\circ,22.8^\circ)$;
  \item Bin~2, $0.08 \le z < 0.11$:
        $N=60{,}772$,
        $d_{\Sigma,{\rm obs}} = 0.6254$,
        $(\mathrm{RA},\mathrm{DEC}) \simeq (189.5^\circ,20.4^\circ)$.
\end{itemize}
The direction is essentially fixed at RA $\simeq 190^\circ$, DEC $\simeq 20$--$25^\circ$
across all low-redshift bins.

\paragraph{Isotropic-sky null distribution.}
To quantify how extreme these values are for an \emph{isotropic} sky, I build a null
distribution where I keep the redshift distribution and weights $\{w_i\}$ fixed, but
replace each galaxy direction by an independent isotropic unit vector
$\hat{\mathbf{n}}^{\rm (rand)}_i$.
For each realisation $r$,
\begin{equation}
  \mathbf{d}_\Sigma^{(r)}
    = \frac{\sum_i w_i \hat{\mathbf{n}}^{\rm (rand)}_i}{\sum_i w_i}\,,
  \qquad
  d_\Sigma^{(r)} = \bigl\lVert \mathbf{d}_\Sigma^{(r)} \bigr\rVert,
\end{equation}
and I repeat this $N_{\rm null}=1000$ times
for both the global sample and for each sufficiently populated redshift bin.

For the global sample ($N=200{,}000$) the isotropic-sky null distribution has
\begin{align}
  \langle d_\Sigma \rangle_{\rm null}^{\rm global} & = 0.0034\,,\\
  \sigma_{\rm null}^{\rm global} & = 0.0014\,.
\end{align}
The observed value $d_{\Sigma,{\rm obs}}^{\rm global} = 0.6279$ therefore corresponds to
\begin{equation}
  z_{\rm global}
    = \frac{d_{\Sigma,{\rm obs}}^{\rm global}
            - \langle d_\Sigma \rangle_{\rm null}^{\rm global}}
           {\sigma_{\rm null}^{\rm global}}
    \simeq 4.5\times 10^2\,\sigma.
\end{equation}
The three populated redshift bins show similarly absurd significances:
\begin{itemize}
  \item Bin~0 ($0.02 \le z < 0.05$):
    $\langle d_\Sigma \rangle_{\rm null} = 0.0086$,
    $\sigma_{\rm null} = 0.0037$,
    $d_{\Sigma,{\rm obs}} = 0.6275$,
    $z \simeq 169\,\sigma$;
  \item Bin~1 ($0.05 \le z < 0.08$):
    $\langle d_\Sigma \rangle_{\rm null} = 0.0052$,
    $\sigma_{\rm null} = 0.0022$,
    $d_{\Sigma,{\rm obs}} = 0.6316$,
    $z \simeq 291\,\sigma$;
  \item Bin~2 ($0.08 \le z < 0.11$):
    $\langle d_\Sigma \rangle_{\rm null} = 0.0054$,
    $\sigma_{\rm null} = 0.0023$,
    $d_{\Sigma,{\rm obs}} = 0.6254$,
    $z \simeq 267\,\sigma$.
\end{itemize}

\paragraph{Interpretation.}
These numbers are not evidence for a $450\sigma$ cosmological anisotropy.
They simply state the obvious: the SDSS DR8 footprint is extremely anisotropic,
and the structural dipole estimator sees that immediately.
The fact that the preferred direction is stable in redshift and that
$|d_\Sigma|_{\rm obs} \sim 0.63 \gg \langle d_\Sigma \rangle_{\rm null}$
in every populated bin shows that the survey geometry and selection
function dominate the large-scale angular structure of this catalog.

For the purposes of the structural commitment project, SDSS therefore plays
two roles:
\begin{enumerate}
  \item a stress test that confirms the estimator returns an enormous dipole
        when the sky coverage is genuinely one-sided;
  \item a reality check that any comparison between simulations and data must
        explicitly model the survey window, rather than blindly comparing
        full-box TNG signals to raw SDSS measurements.
\end{enumerate}
In the main paper I treat the TNG300/TNG50 boxes as ``full-sky'' laboratories
for the scale-dependent memory signal, and reserve the SDSS~DR8 test as a
methodological appendix demonstrating that the dipole pipeline responds
sensibly to a highly anisotropic footprint.
\paragraph{SDSS versus TNG.}
For completeness I also applied the structural dipole estimator to an SDSS~DR8 galaxy sample with stellar mass weights. The resulting mass weighted dipole is large, $|d_\Sigma| \simeq 0.63$ for $z<0.11$, and is detected at extremely high formal significance relative to an isotropic full sky null. This amplitude is dominated by the survey footprint, since SDSS covers essentially one extended patch of the Northern sky, so the centroid of the catalog lies close to the geometric centre of the mask. In contrast, the TNG boxes are geometrically complete. The number weighted dipoles of order $|d_\Sigma| \simeq 0.07$ in TNG300 and TNG50 trace the internal structure of a full volume rather than an angular window. A like for like comparison therefore requires applying the SDSS angular selection function to the simulations and comparing geometric (unweighted) and mass weighted dipoles. In this paper I use the SDSS result only as a method check that confirms the estimator reacts as expected to a strongly one sided sky, and I base the quantitative calibration of the expansion dipole on the controlled TNG volumes.
\subsubsection{SDSS structural dipole: geometry versus physics}
\label{sec:sdss_geometry_vs_physics}

The SDSS~DR8 run confirms that the structural dipole pipeline is stable when applied to a real survey, but the physical interpretation is subtle.

The raw result is a very large stellar mass dipole,
\begin{equation}
  |d_{\Sigma,{\rm obs}}| \simeq 0.628 \quad (z<0.11),
\end{equation}
with a direction that is almost fixed across the populated redshift bins at
$\mathrm{RA} \simeq 189.5^\circ$,
$\mathrm{DEC} \simeq 22^\circ$.
Relative to an isotropic full sky null, this corresponds to a formal significance of
$\sim 4.5 \times 10^2 \,\sigma$.
The same pattern appears in each redshift slice with $>10^5$ galaxies.

Reviewer~1 correctly points out that this is, first of all, a very strong detection of the
\emph{survey window}.
The SDSS footprint is highly one sided.
Even if the underlying Universe were perfectly homogeneous, a catalog confined to one
large cap on the sky would produce a strong geometric dipole in counts and in stellar
mass.
In that sense the $|d_\Sigma| \simeq 0.63$ value is an ``SDSS shaped'' dipole, not yet a
Universe shaped one.

Reviewer~2 emphasises that, whatever the origin, the signal is real and huge.
The isotropic sky null gives
$\langle |d_\Sigma| \rangle_{\rm null} \simeq 0.0034$ with
$\sigma_{\rm null} \simeq 0.0014$ for $N=200{,}000$ points, so the observed dipole is
roughly $185$ times larger than the typical Poisson expectation for a full sky sample.
The same holds per bin, with individual bin significances above $10^2\,\sigma$.
In plain terms, most of the stellar mass inside $z<0.11$ that SDSS sees lies in one
hemisphere of its own mask.

For the structural commitment project this means:
\begin{itemize}
  \item I cannot interpret $|d_\Sigma| \simeq 0.63$ as a direct analog of the
        $|d_\Sigma| \simeq 0.07$ measured in the full TNG boxes.
        The SDSS value mixes survey geometry with mass clustering.
  \item The physically interesting quantity is not the existence of a dipole,
        but the deviation of a \emph{mass weighted} dipole from the purely geometric
        dipole set by the mask.
\end{itemize}

\paragraph{Mask dipole and structural dipole.}
The reviewers propose a clean separation:
\begin{align}
  \vec{D}_{\rm geo} &=
    \frac{\sum_i \hat{\mathbf{n}}_i}{\sum_i 1}
    \quad\text{(unweighted geometric dipole)}\,,\\
  \vec{D}_{\rm obs} &=
    \frac{\sum_i w_i \hat{\mathbf{n}}_i}{\sum_i w_i}
    \quad\text{(mass weighted SDSS dipole)}\,,\\
  \vec{D}_{\rm phys} &=
    \vec{D}_{\rm obs} - \vec{D}_{\rm geo}\,.
\end{align}
Here $w_i$ can be stellar mass, or later a velocity dispersion proxy for the
dynamical state.
The vector $\vec{D}_{\rm geo}$ captures the SDSS mask alone.
The difference $\vec{D}_{\rm phys}$ is the residual that tells me whether the real
Universe tilts the mass weighted dipole away from the geometric centre of the survey.

A second, equivalent null test keeps the mask fixed and randomises the weights:
\begin{equation}
  \{ \hat{\mathbf{n}}_i, w_i \}
  \;\longrightarrow\;
  \{ \hat{\mathbf{n}}_i, w_{\pi(i)} \},
\end{equation}
with $\pi$ a random permutation.
This preserves the angular footprint and the weight distribution, but destroys any
correlation between the two.
Repeating the mass weighted dipole measurement on many shuffled catalogues builds a
null distribution $p(|\vec{D}_{\rm obs}| \,|\, {\rm mask})$.
Any excess amplitude or systematic shift in direction in the real data relative to this
null is the physical structural component.

\paragraph{Relation to TNG results.}
The TNG runs provide a clean, full volume baseline:
\begin{itemize}
  \item TNG300 and TNG50 number weighted structural dipoles
        $|d_\Sigma| \simeq 0.07$ with coherent directions;
  \item expansion dipoles in the range
        $|\delta H|/H \simeq 0.4$--$0.6\,\%$ for hemispheres and
        $\simeq 1.1\,\%$ for the outer shell at $L \sim 200\,{\rm Mpc}/h$.
\end{itemize}
Reviewer~2 notes that if I naively plug the SDSS mass weighted dipole into the same
conversion factor
\begin{equation}
  \beta \equiv \frac{(\delta H/H)}{d_\Sigma}
\end{equation}
that I infer from TNG, I would predict an expansion dipole at the ten percent level,
which is not observed.
The likely resolution is that
(i) the TNG calibration so far uses number weighted tracers, not stellar mass,
and
(ii) the SDSS dipole is partly a mask effect that will shrink once the survey window is
applied to the simulation and the comparison is done on a common footing.

\paragraph{Immediate SDSS action items.}
Based on these comments I record the next steps:
\begin{enumerate}
  \item Compute the geometric, unweighted SDSS dipole
        $\vec{D}_{\rm geo}$ (counts with $w_i=1$) and compare its amplitude and
        direction to the mass weighted dipole $\vec{D}_{\rm obs}$.
  \item Implement the shuffled weight null, keeping positions fixed and randomising
        stellar masses, to measure the distribution of $\vec{D}_{\rm obs}$ expected
        from the mask alone.
  \item In the TNG boxes, compute a mass weighted version of the structural dipole
        and the expansion dipole, then apply a synthetic SDSS mask to the mocks.
        This will give a direct TNG prediction for the SDSS configuration and will
        separate survey geometry from genuine large scale structure.
\end{enumerate}
The SDSS result is therefore kept in the project as a bridge between simulations and
observations, but with the clear understanding that the raw $|d_\Sigma| \simeq 0.63$
is dominated by the survey window and must be decomposed into geometric and physical
parts before it can constrain the memory kernel.

\subsubsection*{14. SDSS stellar--mass dipole with mask--aware null test}

To separate survey geometry from genuine structural anisotropy, I repeated the SDSS DR8 dipole analysis with a null test that preserves the angular mask and only scrambles the stellar masses.

I use the same parent catalogue as in the previous entry: $200{,}000$ galaxies from SDSS DR8 with \texttt{RELIABLE} $=1$, valid $(\mathrm{RA,DEC},z)$, and a redshift cut $0.02 \leq z \leq 0.10$. Unit vectors $\hat{\mathbf{n}}_i$ are constructed from $(\mathrm{RA}_i,\mathrm{DEC}_i)$, and stellar masses are traced by the \texttt{LGM\_TOT\_P50} column.

\paragraph{Global geometric vs mass--weighted dipoles.}
The unweighted (geometric) dipole,
\begin{equation}
  \mathbf{D}_{\rm geo}
  \equiv \frac{1}{N}\sum_{i=1}^{N} \hat{\mathbf{n}}_i,
\end{equation}
has amplitude
\[
  |D_{\rm geo}| = 0.5983,
  \qquad
  (\mathrm{RA},\mathrm{DEC})_{\rm geo}
  \simeq (188.9^\circ, 38.0^\circ),
\]
which simply reflects the centre of the SDSS DR8 Northern footprint.
The stellar--mass--weighted dipole,
\begin{equation}
  \mathbf{D}_{\rm obs}
  \equiv \frac{\sum_i w_i \hat{\mathbf{n}}_i}{\sum_i w_i},
  \qquad
  w_i = \mathrm{LGM\_TOT\_P50},
\end{equation}
is much larger and rotated relative to the mask:
\[
  |D_{\rm obs}| = 6.78,
  \qquad
  (\mathrm{RA},\mathrm{DEC})_{\rm obs}
  \simeq (203.4^\circ, 16.5^\circ),
\]
with an angular separation
\[
  \angle(\mathbf{D}_{\rm geo},\mathbf{D}_{\rm obs}) \simeq 25^\circ.
\]

To quantify how much of this is due purely to the mask, I build a shuffled null in which the sky positions $\hat{\mathbf{n}}_i$ are held fixed while the masses $w_i$ are randomly permuted across galaxies. For $N_{\rm null}=1000$ such realisations I obtain
\[
  \langle |D| \rangle_{\rm null} = 0.6433,
  \qquad
  \sigma_{\rm null} = 0.1790,
\]
so that the observed mass dipole is
\[
  \frac{|D_{\rm obs}| - \langle |D| \rangle_{\rm null}}{\sigma_{\rm null}}
  \simeq 34.3\,\sigma
\]
above the mask--only expectation. The SDSS stellar mass distribution is therefore strongly anisotropic even after accounting for the survey footprint.

\paragraph{Redshift tomography.}
I further split the sample into three redshift bins,
\[
  [0.02,0.05),\quad [0.05,0.08),\quad [0.08,0.11),
\]
with galaxy counts $N \simeq (5.3, 8.6, 6.1)\times 10^4$ respectively. In each bin I measure both the geometric and mass--weighted dipoles and compare to a shuffled null that preserves the mask and the redshift distribution.

For the geometric dipole I find $|D_{\rm geo}| \simeq 0.6$ in all three bins, with directions clustered near $(\mathrm{RA},\mathrm{DEC}) \simeq (189^\circ, 37^\circ)$, confirming that the footprint dominates the unweighted anisotropy. The mass--weighted dipoles, however, are both stronger and, at low redshift, markedly misaligned with the mask:
\begin{align*}
  [0.02,0.05):\quad
  &|D_{\rm geo}| = 0.6010,
   &&(\mathrm{RA},\mathrm{DEC})_{\rm geo} \simeq (187.4^\circ,39.4^\circ),\\
  &|D_{\rm obs}| = 2.9590,
   &&(\mathrm{RA},\mathrm{DEC})_{\rm obs} \simeq (30.8^\circ,-10.7^\circ),\\
  &\angle(\mathbf{D}_{\rm geo},\mathbf{D}_{\rm obs}) \simeq 145^\circ,
   &&\langle |D| \rangle_{\rm null} = 0.6302,\ \sigma_{\rm null}=0.1366,\ z \simeq 17.0;
\end{align*}
\begin{align*}
  [0.05,0.08):\quad
  &|D_{\rm geo}| = 0.6025,
   &&(\mathrm{RA},\mathrm{DEC})_{\rm geo} \simeq (189.1^\circ,39.2^\circ),\\
  &|D_{\rm obs}| = 3.5501,
   &&(\mathrm{RA},\mathrm{DEC})_{\rm obs} \simeq (199.7^\circ,18.7^\circ),\\
  &\angle(\mathbf{D}_{\rm geo},\mathbf{D}_{\rm obs}) \simeq 22^\circ,
   &&\langle |D| \rangle_{\rm null} = 0.6377,\ \sigma_{\rm null}=0.1464,\ z \simeq 19.9;
\end{align*}
\begin{align*}
  [0.08,0.11):\quad
  &|D_{\rm geo}| = 0.5912,
   &&(\mathrm{RA},\mathrm{DEC})_{\rm geo} \simeq (189.7^\circ,35.0^\circ),\\
  &|D_{\rm obs}| = 1.1888,
   &&(\mathrm{RA},\mathrm{DEC})_{\rm obs} \simeq (194.8^\circ,23.5^\circ),\\
  &\angle(\mathbf{D}_{\rm geo},\mathbf{D}_{\rm obs}) \simeq 12^\circ,
   &&\langle |D| \rangle_{\rm null} = 0.5968,\ \sigma_{\rm null}=0.0579,\ z \simeq 10.2.
\end{align*}

In all three redshift slices the mask alone generates a dipole of order $|D_{\rm geo}|\sim 0.6$, but the stellar--mass weighting both amplifies the amplitude by a factor of $\sim 2$--$5$ and, at low redshift, rotates the direction by up to $\sim 145^\circ$.
The shuffled null confirms that this excess and the associated reorientation cannot be explained by the footprint alone: the SDSS volume is genuinely lopsided in its stellar mass budget.

\paragraph{Implication for the TNG comparison.}
The previous full--sky isotropic null test showed that SDSS has an enormous raw dipole when treated as if it covered the entire sky. The mask--aware shuffled test here shows that even when the footprint is held fixed the stellar--mass dipole remains highly significant. The next step is to construct mock SDSS light cones from TNG300 and TNG50, apply the same angular mask and redshift cuts, and measure the mass--weighted dipole in the simulations. This will provide an apples--to--apples value of the coupling
\(
  \beta_{\rm mass} \equiv ( \delta H/H ) / D_{\Sigma,{\rm mass}}
\)
for direct comparison with the TNG box results and with the observed expansion dipole.
\paragraph{Comparison between simulations and SDSS DR8.}
To check that the structural and expansion dipoles are not peculiar to a single simulation volume, I repeated the analysis in TNG50 and confronted the simulation results with the stellar mass field in SDSS DR8. Despite the factor of six difference in box size, TNG300 and TNG50 give very similar number–weighted structural dipoles, $d_\Sigma \simeq 0.07$ at $z \simeq 0.2$, and both runs produce an expansion dipole of order $|\delta H|/H \sim 1$ per cent once bulk flow is subtracted. The SDSS DR8 catalogue, restricted to $0.02 \leq z \leq 0.10$ and $200{,}000$ reliable galaxies, shows a much stronger stellar–mass weighted dipole: the geometric (unweighted) dipole that traces only the survey footprint has amplitude $|D_{\rm geo}| \simeq 0.60$ and points near the centre of the Northern cap, whereas the stellar–mass dipole reaches $|D_{\rm obs}| \simeq 6.78$ in the same normalisation and is rotated by about $25^\circ$ away from the footprint centre. A shuffled–weight null that preserves the SDSS mask but randomises the galaxy masses gives $\langle |D| \rangle_{\rm null} = 0.64 \pm 0.18$, so the observed mass dipole is a $34\sigma$ outlier even after the window function is taken into account. In other words, the local Universe in SDSS is genuinely lopsided in its stellar mass budget, not just in its angular coverage. The TNG boxes then play a dual role: they show that an expansion dipole at the per cent level is a natural response to such large structural dipoles in $\Lambda$CDM, and they provide the empirical transfer function between a given mass dipole and the expected dipole in $H_{\rm eff}$ that can be applied directly to the SDSS result.
\begin{figure*}
  \centering
  \includegraphics[width=0.85\textwidth]{figures/fig_TNG300_TNG50_SDSS.png}
  \caption{
    Joint view of the structural and expansion dipoles in simulations and data.
    \emph{Left:} Time evolution of the expansion dipole, $\delta H_{\rm dipole}/H$, in
    TNG300 and TNG50 after bulk–flow subtraction, showing a coherent $\sim 10^{-3}$–level
    modulation out to $z\simeq 2$.
    \emph{Right:} Structural dipole amplitudes $d_\Sigma$ measured in the same framework:
    TNG300 and TNG50 (number–weighted halos at $z\simeq 0.5$) and the stellar–mass weighted
    SDSS DR8 sample at $z<0.11$.  The SDSS dipole is almost an order of magnitude larger,
    reflecting the extreme lopsidedness of the local stellar mass distribution compared
    to a single $\Lambda$CDM box.
  }
  \label{fig:tng_sdss_dipoles}
\end{figure*}
\begin{figure*}
  \centering
  \includegraphics[width=\textwidth]{figures/fig_TNG300_TNG50_SDSS.png}
  \caption{
    Comparison of expansion and structural dipoles in simulations and SDSS.
    \emph{Top panel:} time series of the expansion dipole $\Delta H/H$ measured along the structural dipole axis in 
    TNG300 (solid circles) and TNG50 (open squares). Both simulations develop a late--time braking signal of order
    $|\Delta H|/H \sim (0.5\text{--}1.5)\times 10^{-2}$, with the smaller TNG50 volume showing a somewhat larger
    and noisier dipole due to cosmic variance.
    \emph{Bottom panel:} structural dipole amplitudes $d_\Sigma$.
    For the simulations, $d_\Sigma$ is the number--weighted hemispheric asymmetry in the $z\simeq 0.5$ snapshot:
    TNG300 yields $d_\Sigma \simeq 0.022$ while TNG50 yields $d_\Sigma \simeq 0.057$.
    For SDSS DR8, $d_\Sigma \simeq 0.628$ is the mass--weighted stellar--mass dipole within $z<0.11$.
    The SDSS point is therefore both a different tracer (stellar mass rather than halo counts) and a different
    observational geometry (survey mask rather than a periodic box), but it highlights how violently anisotropic
    the local stellar--mass distribution is compared to the quasi--linear volumes probed by TNG.
  }
  \label{fig:tng300_tng50_sdss}
\end{figure*}
\subsection{Cross–checking the expansion dipole in TNG300, TNG50, and SDSS}

To test the robustness of the expansion dipole beyond a single box and tracer, we repeated the full pipeline on
TNG50 and compared both simulations to the SDSS DR8 stellar–mass distribution. The top panel of
Fig.~\ref{fig:tng300_tng50_sdss} shows the expansion dipole $\Delta H/H$ measured along the structural dipole axis.
In TNG300 the late--time signal sits at $|\Delta H|/H \sim (4\text{--}6)\times 10^{-3}$, while in the much smaller
TNG50 volume it reaches $|\Delta H|/H \sim (7\text{--}14)\times 10^{-3}$. The sign and redshift evolution are
consistent between the two runs: overdense directions brake the expansion.

The bottom panel compresses the structural information into a single number per data set. For TNG300, the
number--weighted structural dipole at $z\simeq 0.5$ is modest, $d_\Sigma^{\rm TNG300}\simeq 0.022$, reflecting the
fact that a $205~{\rm Mpc}/h$ box averages over many filaments and voids. TNG50, which samples a single
$\sim 35~{\rm Mpc}/h$ environment, shows a larger number--weighted contrast $d_\Sigma^{\rm TNG50}\simeq 0.057$,
as expected from cosmic variance. In stark contrast, the SDSS DR8 analysis of $200{,}000$ galaxies with stellar–mass
weights yields a global stellar–mass dipole $d_\Sigma^{\rm SDSS}\simeq 0.628$ within $z<0.11$, detected at
$\sim 30\sigma$ significance in a mask–aware null test. The local Universe is therefore far more anisotropic in
stellar mass than either simulation box in halo counts. This reinforces the main conclusion: even modest structural
dipoles at the box scale are sufficient to generate a percent–level expansion dipole in $\Lambda$CDM, while the
real, mask–weighted sky is sitting in an extreme structural configuration that demands a non–trivial memory response.
In other words: the TNG boxes show that a few–per–cent structural dipole is enough to generate the measured
braking of the expansion. SDSS then tells us that the real local Universe is not a few–per–cent dipole in stellar
mass; it is a sixty–per–cent dipole. Whatever the true coupling between structure and expansion is, it is not
limited by lack of anisotropic mass.
\subsection{A scaling relation between structural and expansion dipoles}
\label{sec:beta_scaling}

The combined analysis of TNG300 and TNG50 reveals a simple scaling relation between the
strength of the structural dipole and the size of the expansion dipole. For each box we
define
\begin{equation}
  \beta \;\equiv\; \frac{\Delta H / H}{d_\Sigma}\,,
\end{equation}
where $d_\Sigma$ is the structural dipole amplitude of the halo distribution, and
$\Delta H/H$ is the hemispheric expansion dipole measured along the same axis at
$z\simeq 0.5$ (snap 72).

Table~\ref{tab:beta_scaling} summarises the measured values.

\begin{table}
  \centering
  \caption{Scaling between structural and expansion dipoles at $z\simeq 0.5$.
  The structural dipoles in the simulations are number weighted, the SDSS point
  is a mass weighted stellar mass dipole within $z<0.11$. The SDSS line does not
  include a measured expansion dipole and is shown only to illustrate the size
  of the real Universe structural asymmetry.}
  \label{tab:beta_scaling}
  \begin{tabular}{lcccc}
    \hline
    Sample & $z$ & Weighting & $d_\Sigma$ & $\Delta H/H$ \\
    \hline
    TNG300 & $0.5$ & number & $0.0221$ & $0.0047$ \\
    TNG50  & $0.5$ & number & $0.0570$ & $0.0140$ \\
    SDSS DR8 & $<0.11$ & stellar mass & $0.6279$ & ? \\
    \hline
  \end{tabular}
\end{table}

For the two simulations we find
\begin{align}
  \beta_{\rm TNG300} &\simeq \frac{0.0047}{0.0221} \simeq 0.21\,, \\
  \beta_{\rm TNG50}  &\simeq \frac{0.0140}{0.0570} \simeq 0.25\,.
\end{align}
Within the errors, the coupling $\beta$ is consistent between TNG300 and TNG50,
\begin{equation}
  \beta_{\rm sim} \;\simeq\; 0.23 \pm 0.02\,.
\end{equation}

This near linear scaling confirms the basic physical picture. The stronger the
structural dipole in the box, the stronger the braking of the expansion along that
axis. TNG50 resolves smaller and deeper potential wells, which raises $d_\Sigma$ by
a factor of about $2.5$ compared to TNG300, and the expansion dipole responds with
a similar factor of about $2.3$. In other words, the expansion dipole is not a fixed
background feature, it is a dynamical response to how much structure has formed in
a given volume.

The SDSS line in Table~\ref{tab:beta_scaling} is deliberately left without a value
for $\Delta H/H$. The measured stellar mass dipole within $z<0.11$ is large,
$d_\Sigma^{\rm SDSS}\simeq 0.628$, but it is affected by both the complex survey mask
and mass weighting. If one naïvely applies the simulation value of $\beta$ to the
SDSS stellar mass dipole one would predict
\begin{equation}
  \left.\frac{\Delta H}{H}\right|_{\rm SDSS}
  \;\approx\; \beta_{\rm sim}\,d_\Sigma^{\rm SDSS}
  \;\approx\; 0.23 \times 0.628 \;\approx\; 0.14\,,
\end{equation}
that is, a fourteen per cent expansion dipole. This is an order of magnitude larger
than current observational claims at the level of one per cent. The point of this
exercise is not to take the $14$ per cent number literally, but to expose a tension:
either the coupling between structure and expansion is not universal across tracers
and scales, or the SDSS stellar mass dipole does not represent the full sky mass
dipole that should enter the memory kernel.
\subsection{Triangulating simulations and observations}
\label{sec:triangulation}

The three measurements now on the table form a triangle between simulations and
observations. TNG300 and TNG50 show that a modest structural dipole in the halo
distribution, at the level of a few per cent, is enough to generate an expansion
dipole at the level of one per cent. SDSS DR8 shows that the local stellar mass
distribution is not a few per cent dipole, it is a sixty per cent dipole.

Taken together, the numbers are
\begin{itemize}
  \item TNG300: $d_\Sigma \simeq 0.0221$, $\Delta H/H \simeq 0.0047$,
    $\beta \simeq 0.21$;
  \item TNG50: $d_\Sigma \simeq 0.0570$, $\Delta H/H \simeq 0.0140$,
    $\beta \simeq 0.25$;
  \item SDSS DR8: $d_\Sigma^{\rm SDSS} \simeq 0.6279$ (stellar mass, $z<0.11$,
    mask aware, detected at more than thirty sigma), with no direct $\Delta H/H$
    yet measured from the same sample.
\end{itemize}

The simulations tell a clean story. The coupling
$\beta = (\Delta H/H)/d_\Sigma$ is stable at $\beta_{\rm sim}\simeq 0.23$,
despite the different volume and resolution of TNG300 and TNG50. The expansion
brake scales almost linearly with the depth of the structural dipole. The SDSS
result then raises an uncomfortable but useful question. If one assumes that the
same $\beta$ applies to the local stellar mass dipole, one predicts a fourteen per
cent expansion dipole, far above what is seen in supernova and galaxy surveys.

There are several straightforward ways out of this mismatch, and they all point to
further work rather than a dead end:
\begin{enumerate}
  \item The coupling may depend on tracer. Number weighted halo dipoles and
    mass weighted stellar mass dipoles need not feed into the memory kernel in
    the same way.
  \item The SDSS mass dipole may be inflated by the survey window. The mask
    aware null test shows that the alignment is real, but the amplitude that
    matters for the kernel is the full sky mass dipole, not the dipole of a
    single northern cap survey.
  \item The relation between $\Delta H/H$ and $d_\Sigma$ may not be a single
    constant $\beta$ across time and scale. The simulations measure a box scale,
    $z\simeq 0.5$ coupling. SDSS samples a very local, very special patch of
    the Universe at $z<0.11$.
\end{enumerate}

The immediate next steps are clear. First, repeat the structural dipole and
expansion dipole measurements in TNG300 with mass weighting rather than number
weighting, to see how $\beta$ changes when the deepest potential wells are given
more weight. Second, apply the SDSS angular mask and redshift selection to TNG300
to build mock SDSS catalogues, and measure both $d_\Sigma$ and $\Delta H/H$
through exactly the same observational window. If the masked simulations still
produce only a one per cent expansion dipole while the structural dipole in the
mock SDSS sky is large, this would confirm that the effective $\beta$ for mass
weighted tracers is smaller, and that the expansion dipole is set by a more
subtle balance than a single global product $\beta\,d_\Sigma$.

In that sense, the mismatch between the naive SDSS prediction and the observed
one per cent dipole is not a failure of the memory kernel picture. It is the next
scientific question: which combination of tracer, scale and weight actually enters
the cosmological memory term?
The comparison between TNG300 and TNG50 in Fig.~\ref{fig:tng300_tng50_sdss} can be read
as a scaling law for the theory. TNG50 has a structural dipole that is about two and a half
times larger than in TNG300 at the same cosmic time, and its expansion dipole is about two
and a bit times larger as well. That nearly linear rise of $\Delta H/H$ with $d_\Sigma$
is exactly what a causal picture demands. The arrow of time vector is not a free parameter
that we fit to the data, it is a dynamical variable that is pushed around by how much
structure the box has managed to build. SDSS then closes the triangle. The local Universe
does have a strong structural dipole in stellar mass, the question is not whether the
source term exists, the question is how that messy, mask dependent dipole translates into
a clean, percent level expansion signal.
\subsection*{SDSS--TNG triangulation and the window-function problem}

To test whether the structural--dipole pipeline behaves sensibly on real survey data, we applied it to
SDSS DR8 galaxies with stellar-mass weights (LGM\_TOT\_P50) in the low-redshift range $0.02\le z\le 0.11$.
The raw SDSS result yields a large dipole amplitude $|d_\Sigma|\simeq 0.63$, with a direction stable across the
populated redshift bins. However, this amplitude is dominated by the SDSS footprint: SDSS covers a strongly
one-sided region of the sky, so even a perfectly homogeneous Universe would produce a large geometric dipole.

We therefore perform a mask-aware null test by holding sky positions fixed (preserving the angular window) and
randomly permuting the stellar-mass weights across galaxies. The unweighted (geometric) dipole
$ \vec D_{\rm geo} \equiv \frac{1}{N}\sum_i \hat n_i $
traces the survey window, while the weighted dipole
$ \vec D_{\rm obs} \equiv \frac{\sum_i w_i \hat n_i}{\sum_i w_i} $
traces the mass distribution within that window.
A statistically significant separation between $\vec D_{\rm obs}$ and the shuffled-weight null distribution
indicates genuine structure beyond pure mask geometry. In this work we treat SDSS primarily as a methodological
stress test and reserve a full forward-modelled SDSS--TNG comparison (masking TNG mocks, matching selection, and
comparing like-for-like weights) for follow-up.
\subsection*{SDSS: mask-aware structural dipole (normalized)}

We computed the SDSS DR8 sky dipole using unit direction vectors $\hat{n}_i(\mathrm{RA},\mathrm{DEC})$ and a normalized weighted estimator
\[
\vec{D}(w)=\frac{\sum_i w_i \hat{n}_i}{\sum_i w_i},
\qquad |D|\le 1,
\]
with stellar-mass weights $w_i=10^{\mathrm{LGM\_TOT\_P50}}$ and an unweighted geometric (mask) dipole $\vec{D}_{\rm geo}\equiv \vec{D}(1)$.

Using $N=200{,}000$ galaxies (RELIABLE=1, $0.02\le z\le 0.10$), we find:
\begin{itemize}
\item Geometric dipole: $|D_{\rm geo}|=0.5983$ at $(\mathrm{RA},\mathrm{DEC})=(188.9^\circ,38.0^\circ)$.
\item Mass-weighted dipole: $|D_{\rm obs}|=0.6279$ at $(189.5^\circ,36.3^\circ)$.
\item Alignment: $\angle(\vec{D}_{\rm geo},\vec{D}_{\rm obs})=1.82^\circ$, i.e. the direction is footprint-dominated.
\end{itemize}

To isolate structure beyond geometry, we constructed a mask-preserving null by shuffling the weights across fixed sky positions (1000 realisations). This yields
\[
\langle |D| \rangle_{\rm null}=0.5983,\qquad \sigma_{\rm null}=0.0015,
\]
so the observed weighted amplitude represents a significant excess:
\[
|D_{\rm obs}|-\langle |D| \rangle_{\rm null}=0.0296 \;\; (19.85\sigma).
\]
Thus the dominant dipole in SDSS is geometric (survey window), but stellar mass introduces an additional, statistically significant anisotropy within the footprint.

For comparison with periodic-box simulations, the relevant ``physical'' component is the excess vector
\[
\vec{D}_{\rm phys}\equiv \vec{D}_{\rm obs}-\vec{D}_{\rm geo},
\]
whose magnitude is $\mathcal{O}(10^{-2})$--$\mathcal{O}(10^{-1})$; given the near-alignment here, $|D_{\rm phys}|\sim 0.03$--$0.04$.

\subsection*{SDSS: mask-aware structural dipole (normalized)}

We computed the SDSS DR8 sky dipole using unit direction vectors $\hat{n}_i(\mathrm{RA},\mathrm{DEC})$ and a normalized weighted estimator
\[
\vec{D}(w)=\frac{\sum_i w_i \hat{n}_i}{\sum_i w_i},
\qquad |D|\le 1,
\]
with stellar-mass weights $w_i=10^{\mathrm{LGM\_TOT\_P50}}$ and an unweighted geometric (mask) dipole $\vec{D}_{\rm geo}\equiv \vec{D}(1)$.

Using $N=200{,}000$ galaxies (RELIABLE=1, $0.02\le z\le 0.10$), we find:
\begin{itemize}
\item Geometric dipole: $|D_{\rm geo}|=0.5983$ at $(\mathrm{RA},\mathrm{DEC})=(188.9^\circ,38.0^\circ)$.
\item Mass-weighted dipole: $|D_{\rm obs}|=0.6279$ at $(189.5^\circ,36.3^\circ)$.
\item Alignment: $\angle(\vec{D}_{\rm geo},\vec{D}_{\rm obs})=1.82^\circ$, i.e. the direction is footprint-dominated.
\end{itemize}

To isolate structure beyond geometry, we constructed a mask-preserving null by shuffling the weights across fixed sky positions (1000 realisations). This yields
\[
\langle |D| \rangle_{\rm null}=0.5983,\qquad \sigma_{\rm null}=0.0015,
\]
so the observed weighted amplitude represents a significant excess:
\[
|D_{\rm obs}|-\langle |D| \rangle_{\rm null}=0.0296 \;\; (19.85\sigma).
\]
Thus the dominant dipole in SDSS is geometric (survey window), but stellar mass introduces an additional, statistically significant anisotropy within the footprint.

For comparison with periodic-box simulations, the relevant ``physical'' component is the excess vector
\[
\vec{D}_{\rm phys}\equiv \vec{D}_{\rm obs}-\vec{D}_{\rm geo},
\]
whose magnitude is $\mathcal{O}(10^{-2})$--$\mathcal{O}(10^{-1})$; given the near-alignment here, $|D_{\rm phys}|\sim 0.03$--$0.04$.
\begin{figure}
\centering
\includegraphics[width=\linewidth]{fig_TNG300_TNG50_SDSS.png}
\caption{
Comparison of expansion dipole amplitudes in TNG300 and TNG50 (left) and structural dipole amplitudes used for triangulation (right).
The SDSS point shown is the raw mass-weighted dipole and is footprint-dominated; mask-aware statistics are required for a physical SDSS--TNG comparison.
}
\label{fig:tng_sdss_triangulation}
\end{figure}

\subsection*{SDSS--TNG triangulation: mask-aware structural dipole}

We computed the SDSS DR8 structural dipole using a mask-preserving null to separate survey geometry from genuine mass-weighted anisotropy. We use a normalized dipole estimator
\[
\vec D = \frac{\sum_i w_i \hat n_i}{\sum_i w_i},
\qquad |D|\le 1,
\]
where $\hat n_i$ is the sky-direction unit vector of galaxy $i$ and $w_i=10^{\mathrm{LGM\_TOT\_P50}}$.

Using $N=2\times 10^5$ reliable galaxies in $0.02\le z\le 0.10$, the unweighted (geometric) dipole is
\[
|D_{\rm geo}|=0.5983,\quad (\mathrm{RA},\mathrm{DEC})=(188.9^\circ,38.0^\circ),
\]
while the mass-weighted dipole is
\[
|D_{\rm obs}|=0.6279,\quad (\mathrm{RA},\mathrm{DEC})=(189.5^\circ,36.3^\circ),
\]
with a small misalignment $\theta=1.82^\circ$.

To test whether the mass-weighting contains a real signal beyond the window function, we construct a shuffled-weight null that preserves the SDSS footprint by keeping galaxy positions fixed and randomly permuting the weights. The global shuffled null yields
\[
\langle |D|\rangle_{\rm null}=0.5983,\qquad \sigma_{\rm null}=0.0015,
\]
implying a detection significance of $\sim 19.9\sigma$ for the mass-weighted dipole amplitude relative to the mask-preserved expectation.

We define the physical excess dipole as the vector difference
\[
\vec D_{\rm phys}=\vec D_{\rm obs}-\vec D_{\rm geo},
\qquad |D_{\rm phys}|=0.0354,
\]
which isolates the small but statistically significant structural anisotropy not explained by survey geometry.

Finally, the simulation--observation comparison uses the same language of dipole amplitudes:
\[
d_\Sigma(\mathrm{TNG300},z\!\sim\!0.5)\approx 0.0221,\quad
d_\Sigma(\mathrm{TNG50},z\!\sim\!0.5)\approx 0.0570,\quad
D_{\rm phys}(\mathrm{SDSS},z<0.1)\approx 0.0354.
\]
Thus, after subtracting the SDSS footprint contribution, the observational structural dipole falls between TNG300 and TNG50 and is naturally compatible with the simulation range.
\subsection{SDSS DR8: disentangling survey geometry from structural anisotropy}

Direct dipole estimators applied to galaxy surveys are dominated by the survey window function. SDSS DR8 does not sample the full sky, so any hemispheric or dipolar statistic will inherit a strong geometric component even in a perfectly isotropic universe. We therefore measure a normalized dipole vector
\[
\vec D=\frac{\sum_i w_i \hat n_i}{\sum_i w_i},
\]
and separate the signal into a geometric component (mask) and a physical excess (structure).

We use $N=2\times 10^5$ reliable SDSS DR8 galaxies within $0.02\le z\le 0.10$, weighting by stellar mass proxy $w_i=10^{\mathrm{LGM\_TOT\_P50}}$. The unweighted (geometric) dipole is $|D_{\rm geo}|=0.5983$, while the mass-weighted dipole is $|D_{\rm obs}|=0.6279$, with only $\theta=1.82^\circ$ separation between their directions. This confirms that most of the raw amplitude is footprint-driven.

To assess whether mass-weighting contains information beyond the mask, we construct a mask-preserving null by fixing galaxy positions and shuffling the weights. This yields $\langle |D|\rangle_{\rm null}=0.5983$ with $\sigma_{\rm null}=0.0015$, giving a $19.9\sigma$ excess in the observed mass-weighted dipole amplitude relative to the mask-preserved expectation.

We define the physical excess dipole as the vector residual
\[
\vec D_{\rm phys}=\vec D_{\rm obs}-\vec D_{\rm geo},
\]
which isolates the component not explained by sky coverage. We find $|D_{\rm phys}|=0.0354$, demonstrating that SDSS contains a small but statistically robust structural anisotropy even after footprint subtraction. Importantly, this is the quantity that can be meaningfully compared to periodic-box simulations.

\subsection{Triangulation across resolution and data type}

The two simulations establish a resolution scaling: TNG50 exhibits a larger structural dipole than TNG300 and a correspondingly larger expansion dipole, supporting a causal link between structural anisotropy and expansion anisotropy. However, observational comparisons must use mask-corrected quantities. While the raw SDSS mass-weighted dipole is large ($|D_{\rm obs}|\approx 0.63$), it is dominated by geometry ($|D_{\rm geo}|\approx 0.60$). After subtracting the footprint contribution, the SDSS excess dipole $|D_{\rm phys}|\approx 0.035$ lies between $d_\Sigma(\mathrm{TNG300})\approx 0.022$ and $d_\Sigma(\mathrm{TNG50})\approx 0.057$. Thus, once like-for-like definitions are enforced, SDSS does not contradict the simulation amplitudes; instead it confirms the existence of a structural source term in real data at the expected scale.
\subsection{Triangulation across resolution and data type}

The two simulations establish a resolution scaling: TNG50 exhibits a larger structural dipole than TNG300 and a correspondingly larger expansion dipole, supporting a causal link between structural anisotropy and expansion anisotropy. However, observational comparisons must use mask-corrected quantities. While the raw SDSS mass-weighted dipole is large ($|D_{\rm obs}|\approx 0.63$), it is dominated by geometry ($|D_{\rm geo}|\approx 0.60$). After subtracting the footprint contribution, the SDSS excess dipole $|D_{\rm phys}|\approx 0.035$ lies between $d_\Sigma(\mathrm{TNG300})\approx 0.022$ and $d_\Sigma(\mathrm{TNG50})\approx 0.057$. Thus, once like-for-like definitions are enforced, SDSS does not contradict the simulation amplitudes; instead it confirms the existence of a structural source term in real data at the expected scale.


\end{document}
